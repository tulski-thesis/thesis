%%%%%%%%%%%%%%%%%%%%%%%%%%%%%%%%%%%%%%%%%%%%%%%%%%%%%%%
%% Bachelor's & Master's Thesis Template             %%
%% Copyleft by Artur M. Brodzki & Piotr Woźniak      %%
%% Faculty of Electronics and Information Technology %%
%% Warsaw University of Technology, 2019-2020        %%
%%%%%%%%%%%%%%%%%%%%%%%%%%%%%%%%%%%%%%%%%%%%%%%%%%%%%%%

\documentclass[
    left=2.5cm,         % Sadly, generic margin parameter
    right=2.5cm,        % doesnt't work, as it is
    top=2.5cm,          % superseded by more specific
    bottom=3cm,         % left...bottom parameters.
    bindingoffset=6mm,  % Optional binding offset.
    nohyphenation=false % You may turn off hyphenation, if don't like.
]{template/eiti-thesis}

\graphicspath{img/}             % Katalog z obrazkami.
\addbibresource{ref.bib} % Plik .bib z bibliografią

\begin{document}

%--------------------------------------
% Strona tytułowa
%--------------------------------------
\instytut{XXXXXX}
\kierunek{XXXXXX}
\specjalnosc{XXXXXX}
\title{Wybrane metody przeprowadzania i detekcji OSINT}
\engtitle{Selected methods of conducting and detecting OSINT}
\author{tulski}
\album{XXXXXX}
\promotor{XXXXXX}
\date{\the\year}
\maketitle

%--------------------------------------
% Streszczenie po polsku
%--------------------------------------
\newpage %\cleardoublepage % Zaczynamy od nieparzystej strony
\streszczenie \lipsum[1]
\slowakluczowe XXX, XXX, XXX

%--------------------------------------
% Streszczenie po angielsku
%--------------------------------------
\newpage
\abstract \lipsum[1]
\keywords XXX, XXX, XXX

%--------------------------------------
% Oświadczenie o autorstwie
%--------------------------------------
\newpage %\cleardoublepage  % Zaczynamy od nieparzystej strony
\pagestyle{plain}
\makeauthorship

%--------------------------------------
% Spis treści
%--------------------------------------
\newpage %\cleardoublepage % Zaczynamy od nieparzystej strony
\tableofcontents

%--------------------------------------
% Rozdziały
%--------------------------------------
%\cleardoublepage % Zaczynamy od nieparzystej strony
\pagestyle{headings}

\newpage

\section{Wstęp}

\subsection{Zdefiniowanie problemu}

\lipsum[1]

\subsection{Cel pracy}

\lipsum[1]

\subsection{Układ pracy}

\lipsum[1]
\newpage % Rozdziały zaczynamy od nowej strony.
\section{Code listings}

% \addmargin pozwala na wcięcie kodu od lewej (tutaj: 6mm).
% Wcięcie pomaga ustawić kod tak, 
% aby numery linii nie były za bardzo na lewo. 
% Druga liczba oznacza wcięcie od prawej. 
\begin{minted}{html}
<html>
  <head>
    <title>Hello world!</title>
  </head>
  <body>
    Hello world!
  </body>
</html>
\end{minted}

\lipsum[2]

\begin{minted}{cpp}
class Collatz {
  private:
    unsigned current_val_;
    void update_val() {
        if( current_val_ % 2 == 0 )
            current_val_ /= 2;
        else
            current_val_ = current_val_ * 3 + 1;
    }

  public:
    explicit Collatz(unsigned initial_value) : 
        current_val_(initial_value) {}
    void print_sequence() {
        unsigned i = 1;
        while( current_val_ > 1 ) {
            std::cout
                << "val " << i << " = " << current_val_
                << std::endl;
            update_val(); ++i;
        }
    }
};

int main() {
  // prints Collatz seqence, starting from 194375
  Collatz seq(194375);
  seq.print_sequence();
  return 0;
}
\end{minted}


%--------------------------------------------
% Literatura
%--------------------------------------------
\newpage %\cleardoublepage % Zaczynamy od nieparzystej strony
\printbibliography

%--------------------------------------------
% Spisy (opcjonalne)
%--------------------------------------------
\newpage
\pagestyle{plain}

% Wykaz symboli i skrótów.
% Pamiętaj, żeby posortować symbole alfabetycznie
\vspace{0.8cm}
\acronymlist
\acronym{PW}{Politechnika Warszawska}

\end{document} % Dobranoc.
