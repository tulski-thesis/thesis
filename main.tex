%%%%%%%%%%%%%%%%%%%%%%%%%%%%%%%%%%%%%%%%%%%%%%%%%%%%%%%
%% Bachelor's & Master's Thesis Template             %%
%% Copyleft by Artur M. Brodzki & Piotr Woźniak      %%
%% Faculty of Electronics and Information Technology %%
%% Warsaw University of Technology, 2019-2020        %%
%%%%%%%%%%%%%%%%%%%%%%%%%%%%%%%%%%%%%%%%%%%%%%%%%%%%%%%

\documentclass[
    left=2.5cm,         % Sadly, generic margin parameter
    right=2.5cm,        % doesnt't work, as it is
    top=2.5cm,          % superseded by more specific
    bottom=3cm,         % left...bottom parameters.
    bindingoffset=6mm,  % Optional binding offset.
    nohyphenation=false % You may turn off hyphenation, if don't like.
]{template/eiti-thesis}
\usepackage{cleveref}

\addbibresource{ref.bib} % Plik .bib z bibliografią

\begin{document}

%--------------------------------------
% Strona tytułowa
%--------------------------------------
\instytut{Telekomunikacji}
\kierunek{Cyberbezpieczeństwo}
\title{Wybrane metody przeprowadzania i detekcji Web Scrapingu}
\engtitle{Selected methods of conducting and detecting of Web Scraping}
\author{Michał Tułowiecki}
\album{304217}
\promotor{dr inż. Mariusz Sepczuk}
\date{\the\year}
\maketitle

%--------------------------------------
% Streszczenie po polsku
%--------------------------------------
\cleardoublepage % Zaczynamy od nieparzystej strony
\streszczenie
Praca koncentruje się na web scrapingu --- technice pozyskiwania informacji z internetu.
Traktuje ją z dwóch perspektyw tj.~jego przeprowadzania oraz zabezpieczania się przed nim.
Przedstawiono teoretyczne podstawy web scrapingu, jego praktyczne zastosowania oraz różne techniki przeprowadzania.
Szczegółowo opisano trzyetapowy proces: pobieranie danych, ich przetwarzanie oraz zapis i prezentację.
Szczególny nacisk położono na metody detekcji web scrapingu, w tym ograniczenie tempa żądań (rate limiting),
wykorzystanie reverse proxy z regułami blokującymi boty oraz identyfikację przy pomocy browser fingerprintingu.
Scraper oraz metody detekcji przedstawiono w postaci studium przypadku, wykorzystując do tego platformę badawczą (sklep internetowy).
Opisane i wdrożone zabezpieczenia przetestowano w kilku scenariuszach: każdą niezależnie oraz ich fuzję.
Zwrócono uwagę na znaczenie web scrapingu jako narzędzia do efektywnego pozyskiwania danych, jak również na istotę skutecznych metod jego detekcji.

\slowakluczowe web scraping, detekcja botów, rate limiting, reverse proxy, browser fingerprinting

%--------------------------------------
% Streszczenie po angielsku
%--------------------------------------
%\newpage
\bigskip
\abstract
The thesis focuses on web scraping --- a technique for extracting information from the Internet.
It considers it from both the perspective of conducting it and securing against it.
The theoretical basis of web scraping, its practical applications and various techniques for conducting it are presented.
Its three-stage process is described in detail: data retrieval, processing, and saving and presentation.
Special emphasis was placed on web scraping detection methods, including rate limiting of requests,
use of reverse proxies with rules to block bots, and browser fingerprinting.
The scraper and detection methods were presented as a case study, using a research platform (an online store).
The described and implemented defenses were tested in various scenarios, both individually and in their conjunction.
Attention was paid to the importance of web scraping as a tool for effective data acquisition, as well as to the essence of effective methods for its detection.
\keywords web scraping, bot detection, rate limiting, reverse proxy, browser fingerprinting

%--------------------------------------
% Spis treści
%--------------------------------------
\cleardoublepage % Zaczynamy od nieparzystej strony

\section*{\contentsname}

\startcontents[mainsections]
\startlist[figures]{lof}
\startlist[tables]{lot}
\printcontents[mainsections]{l}{1}{\setcounter{tocdepth}{3}}

%--------------------------------------
% Rozdziały
%--------------------------------------
\cleardoublepage % Zaczynamy od nieparzystej strony
\pagestyle{headings}
\newpage


\section{Wstęp}\label{sec:wstep}

\subsection{Motywacja}\label{subsec:motywacja}

Sieć WWW (World Wide Web) w swoim pierwotnym założeniu została zaprojektowana z myślą o odczycie przez ludzi.
Od momentu powstania gwałtownie się rozrastała i tendencja ta wciąż się utrzymuje (zob. \autoref{fig:statista-data-volume-in-exabytes-per-month}).
Zważając na jej obecnie ogromny wolumen danych, ograniczone zasoby ludzkie pozwalają na przetworzenie jedynie niewielkiego ich ułamka.
W obliczu powyższych wyzwań niezbędna stała się więc technika web scrapingu polegająca na zbieraniu, ustrukturyzowaniu i przetworzeniu danych, tak aby umożliwić maszynom ich analizę.

\begin{figure}[H]
    \centering
    \captionsetup{width=.8\linewidth}
    \includegraphics[width=\textwidth]{img/statista-data-volume-in-exabytes-per-month}
    \caption
        [Wolumen danych w globalnym konsumenckim ruchu IP w latach 2017–2022]
        {Wolumen danych w globalnym konsumenckim ruchu IP\\ w latach 2017–2022 (w eksabajtach na miesiąc)\\Źródło: \citetitle*{statista-data-volume}~\cite{statista-data-volume}}
    \label{fig:statista-data-volume-in-exabytes-per-month}
\end{figure}

Technika web scrapingu wciąż zyskuje na popularności stając się nieodzownym narzędziem w pracy z danymi.
Pozwala ona na szybkie i efektywne zbieranie ogromnych ilości danych.
Wykorzystuje się ją w wielu dziedzinach --- od badań społecznych, przez analizę konkurencji, po zbieranie danych do uczenia maszynowego i sztucznej inteligencji.
Skala zastosowania scraperów jest imponująca, co potwierdzają dane statystyczne.
W 2020 roku aż 37,2\% ruchu internetowego generowane było przez boty, w tym właśnie przez scrapery (zob. \autoref{fig:bot-traffic}).
Ta statystyka rzuca światło na ogromną rolę, jaką automaty internetowe odgrywają w cyfrowym ekosystemie.
Przy tak znacznej obecności botów, web scraping nie jest już tylko opcją, ale koniecznością dla tych, którzy chcą pozostać konkurencyjni w szybko rozwijającym się świecie danych.

\begin{figure}[H]
    \centering
    \captionsetup{width=.8\linewidth}
    \includegraphics[width=\textwidth]{img/bot-traffic}
    \caption
        [Procentowy rozkład ruchu generowany w internecie przez: złe boty, dobre boty i ludzi w roku 2019]
        {Procentowy rozkład ruchu generowany w internecie przez:\\ złe boty, dobre boty i ludzi w roku 2019\\Źródło: \citetitle*{bot-traffic}~\cite{bot-traffic}}
    \label{fig:bot-traffic}
\end{figure}

Choć zastosowanie web scrapingu ma wiele zalet to wiąże się również z pewnymi zagrożeniami.
Firmy wykorzystują go do budowania przewagi konkurencyjnej, równocześnie dążąc do ochrony własnych danych przed podobnymi działaniami ze strony konkurencji.
W związku z tym, równolegle do rozwoju technik scrapingu rozwijane są metody jego detekcji i zapobiegania.

\subsection{Cel pracy}\label{subsec:cel-pracy}

Niniejsza praca posiada dwa główne cele.
Pierwszym jest analiza i praktyczne wykorzystanie web scrapingu.
Praca dąży do przedstawienia, zbudowania i przetestowania scrapera.
Drugim celem jest opracowanie i wdrożenie kilku różnych metod skutecznie blokujących web scraping.
Oba te cele mają pokazać zarówno perspektywę aktora przeprowadzającego scraping, jak i broniącego się przed nim.

\clearpage

\subsection{Struktura pracy}\label{subsec:struktura-pracy}

Praca ma charakter praktyczny i traktuje web scraping z dwóch perspektyw tj.~jego przeprowadzania oraz zabezpieczania się przed nim.
Składa się z 8 rozdziałów.
\begin{enumerate}[label={\textbf{Rozdział \arabic*}},labelindent=\parindent, leftmargin=*]
    \item \nameref{sec:teoria} stanowi teoretyczny wstęp do tematyki pracy.
    \item \nameref{sec:przeglad-rozwiazan} zawiera opis wybranych metod detekcji web scrapingu.
    \item \nameref{sec:projekt-platformy} koncentruje się na wdrożeniu sklepu internetowego tulski, który posłużył jako platforma do badań i testowania.
    \item \nameref{sec:projekt-scrapera} przedstawia proces projektowania i tworzenia scrapera pobierającego dane z sklepu tulski.
    \item \nameref{sec:wdrozenie-metod-detekcji} koncentruje się na praktycznym zastosowaniu metod detekcji web scrapingu, omawiając ich wdrożenie w kontekście stworzonego sklepu internetowego.
    \item \nameref{sec:wykorzystane-narzedzia} to prezentacja narzędzi i technologii wykorzystanych w pracy, w tym te służące do budowy sklepu, scrapera, jak również do wdrożenia zabezpieczeń.
    \item \nameref{sec:testy} przedstawia wyniki przeprowadzonych testów scrapera i sklepu wraz z zabezpieczeniami, a także dokonuje ich analizy.
    \item \nameref{sec:podsumowanie} podsumowuje główne tematy pracy, wnioski z badań i testów, a także zwraca uwagę na potencjalne kierunki dalszych badań w obszarze web scrapingu i jego detekcji.
\end{enumerate}

\newpage


\section{Web Scraping}\label{sec:teoria}

Web Scraping, znany również jako Web Data Extraction, Web Data Scraping, Web Harvesting i Screen scraping, to technika pozyskiwania informacji z zasobów WWW (World Wide Web)\cite{Zhao2017}.
Proces ten, choć niektóre źródła dopuszczają stosowanie metod manualnych\cite{applications-and-tools}, najczęściej wykorzystuje automatyzację za pomocą dedykowanego oprogramowania.
Zautomatyzowanie web scrapingu, z wykorzystaniem takich narzędzi, umożliwia efektywne i szybkie pozyskiwanie ogromnych ilości danych, liczących setki tysięcy, miliony, a nawet miliardy rekordów.

Sieć WWW jest największym źródłem wiedzy i danych w historii ludzkości.
W swoim pierwotnym założeniu została zaprojektowana z myślą o odczycie przez ludzi.
Od momentu powstania gwałtownie się rozrastała i tendencja ta wciąż się utrzymuje (zob. \autoref{fig:statista-data-volume-in-exabytes-per-month}).
Zważając na jej obecnie ogromny wolumen danych, ograniczone zasoby ludzkie pozwalają na przetworzenie jedynie niewielkiego ich ułamka.
Zatem, w obliczu powyższych wyzwań, niezbędna stała się technika web scrapingu polegająca na zbieraniu, ustrukturyzowaniu i przetworzeniu danych, tak aby umożliwić maszynom ich analizę.

\begin{figure}[H]
    \centering
    \captionsetup{width=.8\linewidth}
    \includegraphics[width=\textwidth]{img/statista-data-volume-in-exabytes-per-month}
    \caption{Wolumen danych w globalnym konsumenckim ruchu IP\newline w latach 2017–2022 (w eksabajtach na miesiąc)\newlineŹródło: \citetitle*{statista-data-volume}~\cite{statista-data-volume}}
    \label{fig:statista-data-volume-in-exabytes-per-month}
\end{figure}

\subsection{Proces Web Scrapingu}\label{subsec:web-scraping-process}

Proces Web Scrapingu, jak przedstawiono na rysunku~\ref{fig:scraping-process}, można podzielić na trzy główne etapy: pobieranie danych, konwersja i przetwarzanie oraz zapis i prezentacja informacji\cite{persson}.

\todo{Popraw ten rysunek}
\begin{figure}[H]
    \centering
    \includegraphics[width=0.8\textwidth]{img/scraping-process}
    \caption{Proces Web Scrapingu}
    \label{fig:scraping-process}
\end{figure}

\subsubsection{Pobieranie danych}

Pierwszy etapem procesu jest pobranie danych zawierających interesujące nas treści.
Zazwyczaj realizuje się to poprzez wysłanie zapytań HTTP (ang. \emph{Hypertext Transfer Protocol}) do jednego lub wielu serwerów WWW\@.
Pobrane, surowe dane zwykle są w formacie HTML (ang. \emph{Hypertext Markup Language}), XML (ang. \emph{Extensible Markup Language}) lub JSON (ang. \emph{JavaScript Object Notation}).

To kluczowy etap z perspektywy cyberbezpieczeństwa, ponieważ to właśnie w nim scraper wchodzi w bezpośrednią interakcję z scrapowaną infrastrukturą.

W związku z zabezpieczeniami stosowanymi przez serwery, ich dostępnością, wydajnością oraz ograniczeniami sieciowymi, etap ten jest zwykle najdłuższym w całym procesie.

\subsubsection{Parsowanie, estrakcja i przetwarzanie danych}

Drugi etap procesu obejmuje operacje przekształcenia wcześniej pobranych danych, takie jak:
\begin{enumerate}
    \item Parsowanie - przekształcenie danych w strukturalną reprezentację łatwiejszą do dalszego przetwarzania
    \item Ekstrakcja - wyodrębnienie interesujących treści
    \item Filtracja - usunięcie niepożądanych i błędnych danych
    \item Mapowanie - przekształcenie danych do pożądanego, jednolitego formatu
\end{enumerate}

\noindent W przypadku scrapowania danych z wielu różnych źródeł, kluczowe jest ich dopasowanie do jednego wspólnego interfejsu, co ułatwia połączenie ich w spójny zbiór danych.

\subsubsection{Zapis i prezentacja informacji}

Ostatni etap Web Scrapingu, obejmujący zapis i prezentację informacji, odgrywa kluczową rolę w udostępnianiu zebranych danych użytkownikom końcowym w przystępnej i zrozumiałej formie\cite{iee-state-of-the-art}.

Dane mogą zostać zapisane w różny sposób, w zależności od potrzeb i preferencji użytkowników docelowych.
Najczęściej stosuje się bazy danych, pliki tekstowe lub arkusze kalkulacyjne, takie jak CSV (ang. \emph{Comma-Separated Values}) czy XLSX (ang. \emph{Microsoft Excel Open XML Spreadsheet}).
Kluczowe jest, aby dane były zapisywane w formatach umożliwiających łatwą przetwarzalność i analizę, co znacznie ułatwia ich dalsze wykorzystanie i analizę\cite{iee-state-of-the-art}.

Prezentacja danych to proces konwersji zebranych informacji na formę wizualną lub tekstową, dostosowaną do łatwego zrozumienia i interpretacji przez użytkowników.
Ten krok może obejmować tworzenie wykresów, tabel, raportów, interfejsów użytkownika oraz innych form wizualizacji, które umożliwiają szybką analizę i wnioskowanie na podstawie dostępnych danych.
Celem jest przedstawienie informacji w klarowny, przystępny sposób, który odpowiada na pytania odbiorców.

\subsection{Metody Web Scrapingu}\label{subsec:web-scraping-methods}

\subsubsection{Kopiuj-Wklej}

Pierwsza i najbardziej podstawowa metoda web scrapingu to kopiuj-wklej.
Jest to proces manualny, polegający na wybieraniu i kopiowaniu danych bezpośrednio z witryn internetowych, a następnie wklejaniu ich do pliku lub bazy danych.
Metoda ta jest czasochłonna i nieefektywna dla dużych ilości danych, ale może być skuteczna w przypadku małych i prostych zadań\cite{state-of-art}.
Nie wymaga specjalistycznych umiejętności programistycznych, co czyni ją dostępną dla szerszej grupy użytkowników.

\subsubsection{Żądania HTTP i parsowanie HTML}

Kolejną metodą scrapowania jest wykonywanie żądań HTTP w celu uzyskania struktury HTML danej strony internetowej.
Po pobraniu, struktura HTML jest poddawana analizie w celu identyfikacji oraz ekstrakcji żądanych elementów, takich jak tekst, obrazy, hiperłącza i inne.
Często wykorzystuje się specjalistyczne biblioteki programistyczne, które umożliwiają wydobywanie poszczególnych elementów, m.in. poprzez zastosowanie wyrażeń XPath.
Metoda ta jest niezbędna w przypadku stron renderowanych statycznie, kiedy to dostęp do uporządkowanych danych z serwera API jest ograniczony lub niemożliwy.

\subsubsection{Żądania HTTP do serwerów API}

Obecnie, w dobie ogromnej popularności frameworków takich jak Angular oraz bibliotek jak React, duża część aplikacji internetowych jest renderowana dynamicznie.
Aplikacje te komunikują się z zapleczem technicznym (backendem) poprzez API (ang. \emph{Application Programming Interface}) wymieniając dane w uporządkowanej formie np. JSON lub XML.
Fakt ten wykorzystuje kolejna metoda scrapowania polegająca na wykonywaniu żądań bezpośrednio do tych interfejsów z ominięciem warstwy wizualnej aplikacji.
Jest to preferowana metoda, ponieważ znacząco redukuje trudności związane z ekstrakcją danych.
Proces ten jest zwykle szybszy i mniej podatny na błędy, które mogą wynikać ze zmian w strukturze HTML@\.
Należy zauważyć, że struktura HTML ulega częstszym zmianom niż kontakt API, co dodatkowo podkreśla efektywność tej metody w stabilnym pozyskiwaniu danych.

\subsubsection{Wykorzystanie przeglądarki internetowej}

Wśród metod web scrapingu wyróżnia się także tę z wykorzystaniem przeglądarki internetowej.
Podstawę jej działania stanowią biblioteki programistyczne udostępniające wysokopoziomowe API przeglądarek internetowych.
W środowisku JavaScript za przykład posłużyć mogą Puppeteer bazujący na Chrome, Chromium i protokole DevTools
oraz Playwright wspierający silniki Chromium, WebKib i Firefox.
Pierwotnym zastosowaniem wspomnianych wyżej narzędzi było testowanie end-to-end, jednak z czasem rozszerzono ich wykorzystanie i funkcje wspierające web scraping.

Metoda ta sprawdzi się szczególnie w scrapowaniu aplikacji internetowych intensywnie wykorzystujących renderowanie JavaScript
lub kiedy wymagane jest środowisko przeglądarki np. w przypadku zapezpieczenia przez Browser Fingerprinting.
Opisywane biblioteki dysponują rozbudowanym wachlarzem funkcji dających pełną kontrolę nad przeglądarką.
Posiadają klasy takie jak strona, myszka, klawiatura i ekran dotykowy z zachowaniami umożliwiającymi nawigowanie po stronach,
ruchy kursorem i pisanie na klawiaturze.

Ponadto specjaliści w dziedzinie web scrapingu tworzą i rozwijają otwarto-źródłowe rozszerzenia wychodzące na przeciw zabezpieczeniom przeciwko scraperom.
Ukrywają ślad (ang. \emph{fingerprint}) wykorzystanych przeglądarek oraz maksymalnie upodobniają ich zachowanie do człowieka e.g. przez powolniejsze i płyniejsze ruchy kursorem.

To wszystko sprawia, że takie scrapery są relatywnie łatwe do stworzenia.
Natomiast, w związku z wykorzystaniem przez nie rozbudowanego kontekstu przeglądarki, wymagają więcej zasobów i są mniej efektywne.

\subsection{Zastosowania Web Scrapingu}\label{subsec:web-scraping-applications}

Web scraping to jedno z najcenniejszych narzędzi w obszarze pracy z danymi, umożliwiając pozyskanie ogromnych ilości danych z niemal nieograniczonych zasobów internetu\cite{Zhao2017}.
Dzięki automatyzacji i łatwemu dostępowi do niemal nieograniczonych zasobów, metoda ta zapewnia efektywne zbieranie danych przy relatywnie niskich kosztach.
Poniżej opisano przykładowe obszary, w których stosuje się web scraping.

\subsubsection{Business Intelligence}
Web Scraping może służyć za narzędzie, które pomaga firmom w podejmowaniu świadomych decyzji biznesowych i budowaniu przewagi konkurencyjnej.
Najczęściej zbieranymi informacjami są informacje o asortymencie konkurencji, cenach, promocjach, dostępności produktów, opinie klientów oraz dane kontaktowe.
Na rynku istnieją firmy, takie jak Doubledata\cite{doubledata}, które sprzedają web scraping konkurencji w formie usługi.
Podczas przeglądania internetu i literatury, można zauważyć, że rynek e-commerce jest jednym z najczęstszych kontekstów, w którym omawiane jest zastosowanie web scrapingu.

\subsubsection{Marketing i PR}

Web Scraping odgrywa istotną rolę w monitorowaniu treści internetowych i mediów społecznościowych, takich jak Twitter, Facebook, Instagram, LinkedIn czy YouTube.
Jest stosowany do śledzenia opinii publicznej, obserwowania trendów oraz monitorowania wzmianek o marce.
Automatyzacja tego procesu umożliwia szybką reakcję, co jest niezwykle ważne w efektywnym zarządzaniu reputacją marki\cite{monitoring-social-media}.

\subsubsection{Machine Learning i Artificial Intelligence}

Naukowcy i inżynierowie wykorzystują web scraping do pozyskiwania danych, które są niezbędne do trenowania i modeli sztucznej inteligencji\cite{openai-data-collection}.
Dane te wykorzystywane są w różnych celach, od automatycznego rozpoznawania obrazu po analizę języka naturalnego.

\subsection{Scraper}

Web Scraper, w uproszczeniu scraper, to specjalny rodzaj bota przeprowadzający zautomatyzowany proces web scrapingu.

Jak wspomniano w \autoref{subsec:web-scraping-process}, pobieranie danych, czyli etap bezpośredniej interakcji scrapera ze scrapowaną infrastrukturą,
jest jedynym momentem w którym możliwa jest jego identyfikacja i zablokowanie.
Wybrane cechy ruchu sieciowego wykonywanego przez tego typu boty kluczowe w ich detekcji to:

\begin{itemize}
    \item nagłówek \mintinline{text}{User-Agent} wskazuje na narzędzie programistyczne - wget, curl, Postman, Axios itd.
%    \item declares its user-agent as being wget, curl, webcopier etc - it's probably a bot.
    \item nie posiadają nagłówka \mintinline{text}{User-Agent},
%    \item no user-agent (or matching a pattern of known bad ones) - it's probably a bot.
    \item nie posiadają ciasteczek ignorując je,
%    \item no cookie, and wont honor a set cookie - it's probably a bot.
    \item rekurencyjne zapytania o coraz bardziej szczegółowe dane,
%    \item requests details -> details -> details -> details ad nauseum - it's probably a bot.
    \item zapytania jedynie zawartość HTML z ominięciem CSS lub JS,
%    \item requests the html, but not .css, .js or site furniture - it's probably a bot.
    \item duża liczba zapytań HTTP z kodami odpowiedzi > 400.
%    \item generates a large number of HTTP error codes > 400 (1.e 401, 403, 404 \& 500)- it's probably a bot.
    \item pochodzą z mało prawdopodobnego źródła ruchu dla ludzi (np.  Amazon AWS),
%    \item originates from an unlikely human traffic source (i.e Amazon AWS) - it's probably a bot.
    \item nigdy nie posiadają nagłówka z adresem odsyłającym - \mintinline{text}{Referer},
%    \item no referrer, ever - it's probably a bot.
    \item sesje z dużą liczbą zapytań\cite{bot-buster}.
%    \item sessions with a lot of hits. it's probably a bot.
\end{itemize}

\newpage


\section{Przegląd istniejących rozwiązań}\label{sec:przeglad-rozwiazan}

\subsection{Rate Limiting}\label{subsec:rate-limiting}

Jak wskazuje \emph{OWASP API Security Top 10} w punkcie \emph{API4:2019 Lack of Resources \& Rate Limiting}:
\begin{displayquote}[\citetitle*{owasp-api-security-top-10}~\cite{owasp-api-security-top-10}, tłum. własne]
    Żądania API zużywają zasoby takie jak sieć, CPU, pamięć i miejsce na dysku.
    Ilość zasobów potrzebnych do zaspokojenia żądania w dużej mierze zależy od danych wejściowych użytkownika i logiki biznesowej koncówki.
    Należy również wziąć pod uwagę fakt, że żądania od wielu klientów API konkurują o te same zasoby.
    API jest podatne na problemy, jeśli brakuje przynajmniej jednego z następujących limitów lub są one ustawione nieodpowiednio (np. zbyt niskie/wysokie):

    \begin{itemize}
        \item Limit czasu wykonania
        \item Maksymalna alokowalna pamięć
        \item Liczba deskryptorów plików
        \item Liczba procesów
        \item Rozmiar ładunku żądania (np. przesyłane pliki)
        \item Liczba żądań na klienta/zasób
        \item Liczba rekordów na stronę zwracanych w pojedynczej odpowiedzi na żądanie
    \end{itemize}
\end{displayquote}

\noindent

\todo{Rate Limiting}

\subsection{Web Application Firewall}\label{subsec:waf}

\todo{WAF}

\subsection{Browser Fingerprinting}\label{subsec:browser-fingerprinting}

\todo{Browser Fingerprinting}

\newpage

\section{Projekt platformy}\label{sec:projekt-platformy}

\todo{Wprowadzenie}

\begin{figure}[p]
    \centering
    \includegraphics[width=\textwidth]{img/main}
    \caption{Projekt platformy}
    \label{fig:platform-model}
\end{figure}

\subsection{Platforma wdrożeniowa Kubernetes}\label{subsec:platforma-wdrozeniowa-kubernetes}

\todo{Platforma wdrożeniowa Kubernetes}

\subsection{Konfiguracja DNS}\label{subsec:konfiguracja-dns}

\todo{Konfiguracja DNS}

\subsection{Ingress Controller}\label{subsec:ingress-controller}

\begin{listing}[H]
    \begin{minted}{bash}
helm install ingress-nginx \
    --version 1.0.2 \
    --set controller.kind="daemonset" \
    --set controller.hostNetwork=true \
    --set controller.ingressClass.name="public" \
    --set controller.service.create=false \
    --set controller.enableCertManager=true \
    -n ingress-nginx \
    --create-namespace \
    oci://ghcr.io/nginxinc/charts/nginx-ingress
    \end{minted}
    \caption{Polecenie instalujące pakiet oci://ghcr.io/nginxinc/charts/nginx-ingress}
    \label{lst:helm-install-ingress-controller}
\end{listing}


\subsection{Monitoring}\label{subsec:monitoring}

\todo{Observability Plane}

\begin{listing}[H]
    \begin{minted}{bash}
helm install observability \
    -f observability/values.yaml \
    --set "grafana.adminPassword=<adminPassword>" \
    -n observability \
    --create-namespace \
    prometheus-community/kube-prometheus-stack
    \end{minted}
    \caption{Polecenie instalujące pakiet prometheus-community/kube-prometheus-stack}
    \label{lst:helm-install-observability}
\end{listing}

\begin{figure}[p]
    \begin{figure}[H]
        \centering
        \includegraphics[width=\textwidth]{img/grafana-kubernetes-networking-cluster-dashboard}
        \caption{Dashboard Kubernetes / Networking / Cluster}
        \label{fig:grafana-kubernetes-networking-cluster-dashboard}
    \end{figure}

    \begin{figure}[H]
        \centering
        \includegraphics[width=\textwidth]{img/grafana-node-explorer-dashboard}
        \caption{Dashboard Node Exporter / Nodes}
        \label{fig:grafana-node-exlorer-dashboard}
    \end{figure}
\end{figure}

\subsection{Cert Manager}\label{subsec:cert-manager}

\begin{listing}[H]
    \begin{minted}{bash}
helm install cert-manager \
    --version v1.13.1 \
    --set installCRDs=true \
    --namespace cert-manager \
    --create-namespace \
    jetstack/cert-manager
    \end{minted}
    \caption{Polecenie instalujące pakiet jetstack/cert-manager}
    \label{lst:helm-install-cert-manager}
\end{listing}

\subsection{Docker Registry}\label{subsec:docker-registry}

\todo{Docker Registry}

\subsection{Store}\label{subsec:store}

\subsubsection{Przestrzeń nazw}

Przestrzenią nazw (ang. \emph{namespace}) nazywamy zbiór znaków (nazw) należących do jednego kontekstu.
Ich stosowanie umożliwia lepszą organizację i izolację zasobów, co przyczynia się do podniesienia bezpieczeństwa.
W środowisku Kubernetes przestrzenie nazw umożliwiają segmentację jednego klastra na mniejsze, logicznie wyizolowane jednostki.
Przestrzenie nazw mogą reprezentować środowiska różnych klientów (\emph{tenants}) lub środowiska na różnych poziomach (np. testowe i produkcyjne).

Opisywana platforma posiada przestrzeń nazw o nazwie \url{store}, która zawiera wszystkie elementy niezbędne do działania sklepu internetowego \url{store.tulski.com}.
Przestrzeń \url{store} została stworzona poprzez zaaplikowanie manifestu (zob. listing~\ref{lst:store-namespace}).

\begin{listing}[H]
    \inputminted[xleftmargin=20pt,linenos]{yaml}{code/store-namespace.yaml}
    \caption{Manifest tworzący przestrzeń nazw store}
    \label{lst:store-namespace}
\end{listing}

\subsubsection{Baza danych}

Dane sklepu internetowego są fizycznie przechowywane na dyskach maszyn wirtualnych.
W CLI MicroK8s aktywowano dodatek Hostpath-Storage (zob. listing~\ref{lst:enable-hostpath-storage}), który udostępnia katalog hosta woluminom Kubernetes tj. PersistentVolumes.

\begin{listing}[H]
    \begin{minted}{bash}
microk8s enable hostpath-storage
    \end{minted}
    \caption{Polecenie aktywujące dodatek Hostpath-Storage}
    \label{lst:enable-hostpath-storage}
\end{listing}

\noindent Bazą danych jest PostgreSQL zainstalowany przy użyciu menadżera pakietów Helm (zob. rozdział~\ref{subsec:helm}).
Poleceniem \autoref{lst:helm-install-store-db} zainstalowano pakiet \url{bitnamicharts/postgresql} w przestrzeni \url{store}.

\begin{listing}[H]
    \begin{minted}{bash}
helm install store-db \
    --set auth.postgresPassword="<postgres-password>" \
    --set auth.username="store-db-admin" \
    --set auth.password="<password>" \
    --set auth.database="store" \
    --set metrics.enabled=true \
    --set metrics.serviceMonitor.enabled=true \
    --set metrics.serviceMonitor.namespace="observability" \
    --set metrics.serviceMonitor.labels.release="observability" \
    -n store \
    oci://registry-1.docker.io/bitnamicharts/postgresql
    \end{minted}
    \caption{Polecenie instalujące pakiet bitnamicharts/postgresql}
    \label{lst:helm-install-store-db}
\end{listing}

\subsubsection{Backend}

Backend, w rozumieniu architektury wielowarstwowej, pełni kluczowy element systemu informatycznego.
To warstwa stanowiąca zaplecze technologiczne, która w oparciu na zdefiniowanych regułach biznesowych, odpowiada za obsługę żądań klientów oraz przetwarza dane w sposób zapewniający ich spójność.

Serwer dostarczany przez projekt Medusa nie wymagał żadnych modyfikacji.
Jedynym wymogiem do jego uruchomienia było wybranie i dołączenie odpowiednich wtyczek (ang. \emph{plugins}) odpowiedzialnych za zarządzanie zamówieniami oraz obsługę płatności.
W tym celu zastosowano odpowiednio medusa-fulfillment-manual  i medusa-payment-manual (zob. \autoref{lst:medusa-config-fulfillment-payment}).
Obie te wtyczki można postrzegać jako atrapy (ang. \emph{mocks}), które umożliwiają funkcjonowanie serwera bez konieczności integracji z rzeczywistymi serwisami, takimi jak Stripe czy PayPal.

\begin{listing}[H]
    \begin{minted}[xleftmargin=20pt,linenos]{js}
const plugins = [
    `medusa-fulfillment-manual`,
    `medusa-payment-manual`,
    // ...
];
    \end{minted}
    \caption{Konfiguracja pluginów medusa-fulfillment-manual i medusa-payment-manual}
    \label{lst:medusa-config-fulfillment-payment}
\end{listing}

Aby backend mógł zostać uruchomiony w środowisku Kubernetes, koniecznie było jego skonteneryzowania.
Proces ten został zrealizowany przy użyciu narzędzia Docker (zob. podrozdział~\ref{subsec:docker}).
W tym celu stworzono plik Dockerfile (zob. listing~\ref{lst:code-dockerfile-backend}), który definiuje wszystkie kroki budowania obrazu kontenera.
Plik Dockerfile rozpoczyna się od określenia obrazu bazowego, czyli \url{node:18-alpine}.
Następnie, w kolejnych etapach \url{deps}, \url{builder} i \url{runner}, są kolejno instalowane zależności NPM, budowany jest kod aplikacji oraz przygotowywane jest środowisko uruchomieniowe serwera.

\begin{listing}[H]
    \inputminted[xleftmargin=20pt,linenos]{docker}{code/Dockerfile.backend}
    \caption{Plik Dockerfile.backend}
    \label{lst:code-dockerfile-backend}
\end{listing}

\subsubsection{Panel administracyjny}

\todo{Panel administracyjny}

\subsubsection{Witryna internetowa}

\todo{Witryna internetowa}

\subsubsection{Populacja danych}

\todo{Populacja danych}
\newpage


\section{Projekt i wykonanie scrapera}\label{sec:projekt-scrapera}

Niniejszy rozdział poświęcono projektowi i wykonaniu scrapera - narzędzia służącego do automatycznego zbierania danych z sklepu internetowego tulski, omówionego w rozdziale~\ref{sec:projekt-platformy}.
Założono, że celem scrapera jest zgromadzenie informacji o wszystkich produktach dostępnych w sklepie internetowym, które mogą zostać wykorzystane do analizy asortymentu i cen produktów.
Kolejnym, istotnym założeniem rozdziału jest podejście typu czarnej skrzynki (ang. \emph{black-box})\cite{sekurak-testy-penetracyjne}.
Brak początkowej wiedzy na temat struktury i mechanizmów scrapowanego serwisu, jakim jest sklep internetowy tulski, nadaje realizm opisanym działaniom, odzwierciedlając warunki, w jakich najczęściej pracują osoby specjalizujące się w web scrapingu.
Poniższa część pracy jest zatem nie tylko technicznym studium przypadku, ale również wglądem w procesy myślowe i metody, które są wykorzystywane w realnych warunkach.

\subsection{Rekonesans}\label{subsec:rekonesans}

Rekonesans rozpoczął proces tworzenia scrapera, skupiając się na zrozumieniu celu, jakim jest sklep internetowy tulski.
Głównym zadaniem rekonesansu było zidentyfikowanie metod dających efektywny dostęp do interesujących danych.
Skupiono się na zrozumieniu mechanizmu ładowania i renderowania treści, strukturze zasobów oraz wykryciu potencjalnych wyzwań, które mogłyby wpłynąć na proces scrapowania.
Rekonesans jest kluczowym etapem w tworzeniu scrapera, ponieważ od niego zależą efektywność, awaryjność i stabilność narzędzia.
Przykładowo, jeśli podczas rekonesansu uda się znaleźć żądania API, które zwracają dane w formacie JSON, narzędzie będzie potencjalnie szybsze i bardziej odporne na zmiany.
Wynika to z faktu, że kontakt API zmienia się z reguły rzadziej niż interfejs graniczny użytkownika.

Kluczowym elementem rekonesansu była analiza ruchu sieciowego za pomocą narzędzi programistycznych wbudowanych w przeglądarkę internetową Firefox.
Po pierwsze, ustalono, że strona korzysta korzysta z proxy Cloudflare, co wskazuje nagłówek \emph{server} (patrz listing~\ref{lst:rekonesans-get-homepage}).
Po drugie, odkryto, że strona została zbudowana z wykorzystaniem frameworka Next, co wskazuje nagłówek \emph{x-powered-by} (patrz listing~\ref{lst:rekonesans-get-homepage}).
Wykorzystanie Next oznacza wykorzystanie React, co może sugerować dynamiczne renderowanie treści.

\begin{listing}[H]
    \begin{minted}[xleftmargin=10pt,linenos]{text}
$ curl -I https://store.tulski.com
HTTP/2 200
date: Sun, 17 Dec 2023 17:09:05 GMT
content-type: text/html; charset=utf-8
cache-control: private, no-cache, no-store, max-age=0,
    must-revalidate
vary: RSC, Next-Router-State-Tree, Next-Router-Prefetch,
    Next-Url, Accept-Encoding
x-powered-by: Next.js
cf-cache-status: DYNAMIC
report-to: {"endpoints":[{"url":"https:\/\/a.nel.cloudflare.com\
    /report\/v3?s=l8mwfqVrt2h%2BXJYVshGNiHz%2FIRwSCEqIgVqxZ8kBhW
    WqkIGQaAuYiOy9sv4JmNEGP1%2B53j1pVTVnm%2BpX2sID4%2BRDooubYk4t
    ibHwMUz8sjuDxll1NrjXEffI6gReu1VJV7JI"}],"group":"cf-nel",
    "max_age":604800}
nel: {"success_fraction":0,"report_to":"cf-nel","max_age":604800}
server: cloudflare
cf-ray: 8370c5927feb666e-AMS
alt-svc: h3=":443"; ma=86400
    \end{minted}
    \caption{Nagłówki odpowiedzi dla strony domowej sklepu tulski}
    \label{lst:rekonesans-get-homepage}
\end{listing}

Po trzecie, w trakcie rekonesansu zauważono żądania do serwera API \emph{api.tulski.com}.
Odkrycie potwierdza hipotezę o dynamicznym renderowaniu treści na stronie.
Serwer API, podobnie jak witryna internetowa, korzysta z usług Cloudflare.
Zidentyfikowano dwa żądania, które zwracają dane o produktach w formacie JSON\@.
Jednym z zidentyfikowanych żądań jest to, które zwraca listę produktów (zob. \autoref{fig:store-get-products-list}), wykorzystując paginację.
W żądaniu rozmiar strony jest określany parametrem \emph{limit}, a pozycję startową strony - parametrem \emph{offset}.

Przeprowadzono testy metodą prób i błędów, aby określić maksymalną liczbę produktów, jaką może zwrócić API w jednej odpowiedzi.
Ustalono, że parametr \emph{limit} nie posiada ścisłej walidacji, co umożliwia pobranie znacznie większej ilości danych niż to, co zwykle pobiera klient aplikacji internetowej.
Jednakże, zauważono, że ograniczeniem podczas pobierania dużej liczby rekordów jest czas otwarcia połączenia.
W przypadkach, gdy próbowano pobrać bardzo dużą liczbę rekordów, na przykład 8500 (zob. Listing~\ref{lst:store-get-products-list-8500}), serwer odpowiedział statusem \emph{504 Gateway Timeout}, wskazując na przekroczenie maksymalnego dopuszczalnego czasu przetwarzania żadania.
Maksymalną liczbą rekordów jaką udało się pobrać bez błędów było 8000 (zob. Listing~\ref{lst:store-get-products-list-8000}).


\begin{figure}[p]
    \begin{figure}[H]
        \centering
        \includegraphics[width=0.8\textwidth]{img/store-get-products-list}
        \caption{Żądanie zwracające listę produktów}
        \label{fig:store-get-products-list}
    \end{figure}
    \begin{listing}[H]
        \begin{minted}[xleftmargin=10pt,linenos]{bash}
$ curl -X GET 'https://api.tulski.com/store/products?limit=8500' \
        -s \
        -o /dev/null \
        -w 'HTTP Code: %{http_code}\nTime Total: %{time_total}s\n'
HTTP Code: 504
Time Total: 61.691521s
        \end{minted}
        \caption{Żądanie 8500 produktów}
        \label{lst:store-get-products-list-8500}
    \end{listing}
    \begin{listing}[H]
        \begin{minted}[xleftmargin=10pt,linenos]{bash}
$ curl -X GET 'https://api.tulski.com/store/products?limit=8000' \
        -s \
        -o /dev/null \
        -w 'HTTP Code: %{http_code}\nTime Total: %{time_total}s\n'
HTTP Code: 200
Time Total: 60.223100s
        \end{minted}
        \caption{Żądanie 8000 produktów}
        \label{lst:store-get-products-list-8000}
    \end{listing}
\end{figure}



\begin{figure}[H]
    \centering
    \includegraphics[width=0.8\textwidth]{img/store-get-product-details}
    \caption{Żądanie zwracające szczegóły produktu}
    \label{fig:store-get-product-details}
\end{figure}

\newpage


\section{Wdrożenie wybranych metod detekcji}\label{sec:wdrozenie-metod-detekcji}

W infrastrukturze sklepu internetowego wdrożono trzy metod detekcji web scrapingu: Bot Blocker Reverse Proxy, Rate Limiting oraz rozwiązanie bazujące na procesie browser fingerprintingu.

\subsection{Bot Blocker Reverse Proxy}\label{subsec:reverse-proxy-impl}

W infrastrukturze sklepu internetowego wdrożono serwer typu reverse proxy zlokalizowany przed backendem (zob. \autoref{fig:backend-reverse-proxy}).

\begin{figure}[H]
    \centering
    \captionsetup{width=.8\linewidth}
    \includegraphics[width=\textwidth]{img/backend-reverse-proxy}
    \caption{Lokalizacja Reverse Proxy w infrastrukturze sklepu}
    \label{fig:backend-reverse-proxy}
\end{figure}

Serwer wykorzystuje oprogramowanie Nginx, które zostało rozszerzone o narzędzie Nginx Ultimate Bad Bot Blocker.
Zgodnie z opisem autorów, Ngnix Ultimate Bad Bot Blocker służy do obrony przed:
złośliwymi botami i złośliwymi agentami (\texttt{User-Agent}),
techniką Clickjacking, techniką Click Re-Directing,
Spam Referrer, Adware, Malware, Ransomware,
złośliwymi adresami IP z systemami anty-DDoS,
techniką wykrywania motywów Wordpress
oraz organizacjami SEO~\cite{nginx-ultimate-bad-bot-blocker}.

\autoref{lst:nginx-bot-blocker-conf} przedstawia konfigurację serwera.
Instrukcje \texttt{include} ładują konfiguracje dostarczane z Nginx Ultimate Bad Bot Blocker.
Serwer nasłuchuje na port 9090 i przekierowuje żądania na port 9000.
Żądania są przekazywane wraz ze wszystkimi nagłówkami.
Dodatkowo, w celu monitoringu i zapewnienia audytowalności, włączono logowanie zdarzeń w niestandardowym formacie.

Wdrożenie w środowisku Kubernetes wymagało konteneryzacji omawianego rozwiązania.
\autoref{lst:nginx-bot-blocker-dockerfile} przedstawia plik Dockerfile zawierający instrukcje budujące obraz \texttt{Bot Blocker Proxy}.
Jako bazę wykorzystano obraz \texttt{ngnix:stable}.
Instalacja Nginx Ultimate Bad Bot Blocker jest realizowana przez skrypt dostarczany przez twórców.

\begin{listing}[p]
    \begin{minted}[xleftmargin=10pt,linenos,breaklines]{nginx}
http {
    log_format  main  '$remote_addr - $remote_user [$time_local] "$request" '
                      '$status $body_bytes_sent "$http_referer" '
                      '"$http_user_agent" "$http_x_forwarded_for"';

    access_log  /var/log/nginx/access.log  main;

    include /etc/nginx/conf.d/botblocker-nginx-settings.conf;
    include /etc/nginx/conf.d/globalblacklist.conf;

    server {
	    listen  9090;
	    listen  [::]:9090;

	    include /etc/nginx/bots.d/ddos.conf;
        include /etc/nginx/bots.d/blockbots.conf;

	    location / {
		    proxy_pass                  http://localhost:9000;
		    proxy_pass_request_headers  on;
		    proxy_buffering             off;
	    }
    }
}
    \end{minted}
    \caption{Plik konfiguracyjny serwera reverse proxy}
    \label{lst:nginx-bot-blocker-conf}
\end{listing}

\begin{listing}[p]
    \begin{minted}[xleftmargin=10pt,linenos,breaklines]{docker}
FROM nginx:stable
RUN apt-get -y update && apt-get -y install wget
RUN wget https://raw.githubusercontent.com/mitchellkrogza/-
    nginx-ultimate-bad-bot-blocker/master/install-ngxblocker -O /usr/local/sbin/install-ngxblocker
RUN chmod +x /usr/local/sbin/install-ngxblocker
WORKDIR /usr/local/sbin
RUN ./install-ngxblocker -x
WORKDIR /
COPY nginx.conf /etc/nginx/
EXPOSE 9090
CMD ["nginx", "-g", "daemon off;"]
    \end{minted}
    \caption{Instrukcje budujące obraz reverse proxy}
    \label{lst:nginx-bot-blocker-dockerfile}
\end{listing}

\newpage

\subsection{Rate Limiting}\label{subsec:rate-limiting-impl}

Kolejnym wdrożonym zabezpieczeniem przed web scrapingiem był API Rate Limiting.
Wykorzystano możliwość uruchomionego wcześniej (zob. \autoref{subsubsec:ingress-controller}) Ngnix Ingress Controller
do tworzenia zasobów polityk (ang. \emph{Policy Resource}) dla zasobów typu \texttt{VirtualServer} i \texttt{VirtualServerRoute}.
Polityki te pozwalają na kontrolę dostępu (ang. \emph{access control}) i rate limiting~\cite{nginx-ingress-controller-policy-resource}.

Wdrożono politykę \emph{store-api-rate-limit-policy} ograniczającą liczbę żądań do 40 na minutę
dla jednego adresu IP (\texttt{binary\_remote\_addr}), z możliwością przekroczenia limitu o 5 (\texttt{burst}).
Pełną konfigurację \emph{store-api-rate-limit-policy} przedstawia \autoref{lst:rate-limiting-policy}.

Polityka została zaaplikowana do VirtualServer, który zarządza komunikacją z backendem (zob. \autoref{fig:backend-rate-limiting}).
Szczegółowa konfiguracja VirtualServer \texttt{backend-vs}, w tym wykorzystanie polityki \texttt{store-api-rate-limit-policy}, zostało przedstawione na \autoref{lst:rate-limiting-virtual-server}.
Widoczne tam sekcje \texttt{action.pass} mówią o tym, co ma się stać z ruchem pasującym do określonej ścieżki (\texttt{path}).
Rate limiting został zaaplikowany dla ścieżek z prefiksem \texttt{/store} (linia 18, \texttt{path: /store}), zatem dla Store API z pominięciem Admin API wykorzystywanym do zarządzania sklepem.


\begin{figure}[H]
    \centering
    \captionsetup{width=.8\linewidth}
    \includegraphics[width=\textwidth]{img/backend-rate-limiting}
    \caption{Polityka Rate Limiting w infrastrukturze sklepu}
    \label{fig:backend-rate-limiting}
\end{figure}

\begin{listing}[p]
    \begin{minted}[xleftmargin=10pt,linenos,breaklines]{yaml}
apiVersion: k8s.nginx.org/v1
kind: Policy
metadata:
  name: store-api-rate-limit-policy
  namespace: store
spec:
  rateLimit:
    rate: 40r/m
    key: ${binary_remote_addr}
    burst: 5
    zoneSize: 10M
    rejectCode: 429
    \end{minted}
    \caption{Manifest \texttt{store-api-rate-limit-policy}}
    \label{lst:rate-limiting-policy}
\end{listing}

\begin{listing}[p]
    \begin{minted}[xleftmargin=10pt,linenos,breaklines]{yaml}
apiVersion: k8s.nginx.org/v1
kind: VirtualServer
metadata:
  name: backend-vs
  namespace: store
spec:
  ingressClassName: public
  host: api.tulski.com
# tls: ...
  upstreams:
    - name: bot-blocker-proxy
      service: bot-blocker-proxy
      port: 9090
  routes:
    - path: /
      action:
        pass: bot-blocker-proxy
    - path: /store
      policies:
        - name: store-api-rate-limit-policy
      action:
        pass: bot-blocker-proxy
    \end{minted}
    \caption{Szczegółowa konfiguracja VirtualServer \texttt{backend-vs}}
    \label{lst:rate-limiting-virtual-server}
\end{listing}

\newpage

\subsection{Browser Fingerprinting}\label{subsec:browser-fingerprinting-impl}

Opracowano i wdrożono rozwiązanie korzystające z procesu browser fingerprintingu mające na celu identyfikację i blokowanie nieautoryzowanego dostępu do zasobów sklepu internetowego tulski przez scrapery.
Zasadę działania systemu przedstawia \autoref{fig:bot-detection-sequence}.

\begin{figure}[H]
    \centering
    \includegraphics[width=.95\textwidth]{img/bot-detection-sequence}
    \caption{Diagram sekwencji rozwiązania wykorzystującego browser fingerprinting}
    \label{fig:bot-detection-sequence}
\end{figure}

Stworzono nową aplikację \texttt{Bot API} oraz bibliotekę \texttt{@tulski/\\bot-client}.
Dodatkowo, zmodyfikowano istniejącą witrynę internetową i backend.

\subsubsection{Bot API}

\texttt{Bot API} to aplikacja webowa napisana w języku TypeScript z wykorzystaniem platformy Express.js.
Środowiskiem uruchomieniowym aplikacji jest Node.js.
Do analizy odcisku przeglądarki wykorzystano bibliotekę \texttt{fpscanner}~\cite{github-fpscanner}.
Browser fingerprint jest poddawany 21 testom (zob. \autoref{lst:fpscanner-tests}), z których każdy
może zwrócić 3 wartości: \texttt{Consistent}, \texttt{Unsure} i \texttt{Inconsistent}.
Odcisk jest klasyfikowany jako pochodzący od złośliwego bota (\texttt{"bad\_bot"}) jeżeli:
\begin{itemize}
    \item więcej niż jeden test zwraca wartość \texttt{Inconsistent},
    \item przynajmniej jeden test zwraca wartość \texttt{Inconsistent} i przynajmniej jeden test zwraca wartość \texttt{Unsure}.
\end{itemize}

\begin{listing}[H]
    \begin{minted}[xleftmargin=10pt,linenos,breaklines]{text}
[
  "PHANTOM_UA", "PHANTOM_PROPERTIES", "PHANTOM_ETSL", "PHANTOM_LANGUAGE", "PHANTOM_WEBSOCKET", "MQ_SCREEN", "PHANTOM_OVERFLOW", "PHANTOM_WINDOW_HEIGHT", "HEADCHR_UA", "WEBDRIVER", "HEADCHR_CHROME_OBJ", "HEADCHR_PERMISSIONS", "HEADCHR_PLUGINS", "HEADCHR_IFRAME", "CHR_DEBUG_TOOLS", "SELENIUM_DRIVER", "CHR_BATTERY", "CHR_MEMORY", "TRANSPARENT_PIXEL", "SEQUENTUM", "VIDEO_CODECS"
]
    \end{minted}
    \caption{Tablica testów na obecność śladu złośliwego bota}
    \label{lst:fpscanner-tests}
\end{listing}


\noindent API udostępnia trzy końcówki.
\begin{enumerate}
    \item \texttt{GET /health} służący do monitorowania stanu aplikacji.\ Wykorzystywany przed Kubernetes w mechanizmie \texttt{Liveness Probe}.
    \item \texttt{POST /api/v1/analyze} przyjmujący na wejściu browser fingerprint, który jest poddawany analizie.
    Zwraca unikalny identyfikator przeprowadzonej analizy.
    \item \texttt{GET /api/v1/verify/:id} dający dostęp do wyniku analizy z wskazanym identyfikatorem (\texttt{id}).
    Zwraca obiekt JSON, który zawiera atrybut \texttt{result} przyjmujący wartość \texttt{"not\_detected"} lub \texttt{"bad\_bot"}.

\end{enumerate}
\noindent Aplikacja została wyposażona w podstawową walidację danych wejściowych.

\subsubsection{@tulski/bot-client}

\texttt{@tulski/bot-client} to biblioteka będąca klientem \texttt{Bot API}.
Biblioteka jest dostępna publicznie jako paczka w NPM Registry pod adresem: \url{https://www.npmjs.com/package/@tulski/bot-client}.
Jej Wykorzystanie polega na stworzeniu instancji klasy \texttt{BotClient}, a następnie wywołaniu metody \texttt{analyze()} zwracającej wynik analizy odcisku przeglądarki.
\texttt{@tulski/bot-client} zbiera browser fingerprint przy pomocy biblioteki \texttt{fp-collect}~\cite{github-fp-collect}.
Interfejs biblioteki przedstawia \autoref{lst:bot-client-types}.

\begin{listing}[H]
    \begin{minted}[xleftmargin=10pt,linenos,breaklines]{typescript}
export declare class BotClient {
    static load(baseUrl: string): Promise<BotClient>;
    constructor(baseUrl: string);
    analyze(): Promise<BotAnalysisResult>;
}
export interface BotAnalysisResult {
    id: string;
    created_at: Date;
    bot: "not_detected" | "bad_bot" | "good_bot";
}
    \end{minted}
    \caption{Definicja typów biblioteki \texttt{@tulski/bot-client}}
    \label{lst:bot-client-types}
\end{listing}

\subsubsection{Witryna internetowa}

W witrynie internetowej zaimplementowano \texttt{BotDetectionProvider}, wykorzystując powszechnie znany wśród specjalistów React wzorzec \texttt{Provider}.
\texttt{BotDetectionProvider} podczas wejścia na stronę:
\begin{enumerate}
    \item Pobiera wynik analizy odcisku przeglądarki z lokalnego magazynu (ang. \emph{local storage}).
    \item Dołącza identyfikator wyniku analizy jako nagłówek \texttt{x-bot-api} do żądań Store API\@.
\end{enumerate}
Gdy local storage nie zawiera wyniku analizy (na przykład przy pierwszym wejściu na stronę), \texttt{Provider} tworzy go wykorzystując bibliotekę \texttt{@tulski/bot-client} i zapisuje w magazynie.

\subsubsection{Backend}

Zmian wymagał również backend sklepu internetowego tulski.
Zaimplementowano interceptor działający dla końcówek \texttt{Store API}, który kolejno:
\begin{enumerate}
    \item Weryfikuje obecność identyfikatora analizy odcisku palca w nagłówku \texttt{x-bot-api}.
    \item Pobiera ten identyfikator.
    \item Komunikuję się z \texttt{Bot API} w celu pobrania pełnego wyniku analizy.
    \item Weryfikuje czy odcisk przeglądarki wskazuje na złośliwego bota.
\end{enumerate}
W przypadku braku nagłówka \texttt{x-bot-api} w żądaniu, serwer zwróci błąd z kodem 403.
Jeśli analiza wykryje złośliwego bota, serwer odpowie kodem błędu 500.

\newpage

\section{Wykorzystane narzędzia}\label{sec:wykorzystane-narzedzia}

W tym rozdziale zaprezentowane zostaną technologie i oprogramowanie, które stanowiły podstawę do stworzenia i zarządzaniem platformy badawczej.

\subsection{Docker}\label{subsec:docker}

Docker to otwarte oprogramowanie do tworzenia, dostarczania i uruchamiania aplikacji\cite{docker-overview}.
Docker umożliwia uruchamianie aplikacji w izolowanym środowisku, które nazywane jest kontenerem.
Kontenery zapewniają izolację, bezpieczeństwo, spójność działania w różnych środowiskach oraz ułatwiają rozwój, testowanie i wdrażanie aplikacji.
Część praktyczna pracy wykorzystuje technologię konteneryzacji oferowaną przez platformę Docker.

\subsection{Kubernetes}\label{subsec:kubernetes}

Kubernetes to przenośna, rozszerzalna, otwarto-źródłowa platforma do zarządzania skonteneryzowanymi jednostkami obciążeniowymi (ang. \emph{workloads}) i usługami, która ułatwia zarówno deklaratywne konfigurowanie, jak i automatyzację\cite{kubernetes-overview}.
Platforma Kubernetes oferuje wiele funkcji, w tym deklaratywne zarządzanie elementami infrastruktury, usługi sieciowe, skalowanie oraz automatyzację wdrażania aplikacji, zapewniając tym samym niezbędne narzędzia potrzebne do budowy infrastruktury systemów informatycznych.

\subsection{MicroK8s}\label{subsec:microk8s}

Kubernetes stał się niejako standardem w branży IT\@.
Wraz z wzrostem jego popularności, kolejne osoby oraz firmy zaczęły dostosowywać oryginalny projekt do swoich potrzeb, tym samym tworząc jego własne dystrybucje.
Chociaż wszystkie te dystrybucje mają wspólny fundament, jakim jest oryginalny projekt Kubernetes, każda z nich wnosi unikalne funkcje, narzędzia i optymalizacje.
Przykładowo, dystrybucje takie jak Amazon Elastic Kubernetes Service (Amazon EKS), Google Kubernetes Engine (GKE) czy Azure Kubernetes Service (AKS) zostały specjalnie dostosowane pod specyficzne środowisko chmury poszczególnych dostawcy.

Niniejsza praca wykorzystuje dystrybucję MicroK8s\cite{microk8s-docs-home} ze względu na:
\begin{enumerate}
    \item łatwość instalacji i uruchomienia,
    \item fakt, że jest to lekka dystrybucja K8s, co przekłada się na mniejsze wymagania sprzętowe oraz mniejsze zużycie zasobów,
    \item stabilność - dystrybucja jest przygotowania do produkcyjnego uruchomienia,
    \item prostotę zarządzania - MicroK8s udostępnia rozbudowany interfejs wiersza poleceń (ang. \emph{command line interface, CLI}) ułatwiający konfigurację i utrzymanie środowiska.
\end{enumerate}

\subsection{Helm}\label{subsec:helm}

Helm to menadżer pakietów w środowisku Kubernetes, który pozwala definiować, instalować i aktualizować najbardziej skomplikowane aplikacje Kubernetes~\cite{helm-home}.
Narzędzie to jest szczególnie przydatne w przypadku aplikacji, które wymagają skomplikowanej konfiguracji i/lub składają się z wielu elementów.

\subsection{PostgreSQL}\label{subsec:postgresql}

PostgreSQL, nazywany także Postgres, to, według twórców, najbardziej zaawansowany otwarto-źródłowy system relacyjnych baz danych\cite{postgresql-home}.

\subsection{NGINX}\label{subsec:nginx}

\todo{NGINX}

\subsection{Cert-Manager}\label{subsec:cert-manager}

Cert-Manager to otwarto-źródłowy orkiestrator certyfikatów X.509 w środowiskach Kubernetes i OpenShift\cite{cert-manager-home}.
Manualny proces tworzenia certyfikatów w środowisku Kubernetes, zaprezentowany w dokumentacji\cite{kubernetes-generate-certificates-manually}, jest skomplikowany i czasochłonny.
Rozwiązaniem tego problemu jest narzędzie Cert-Manager, które go upraszcza, automatyzując proces tworzenia, zarządzania i odnawiania certyfikatów.

%\subsection{Registry}\label{subsec:distribution}
%
%Registry to narzędzie służące do przechowywania i dystrybucji obrazów kontenerów i innych artefaktów, oparte na specyfikacji OCI Distribution\cite{oci-distribution-specification-github}.
%Registry jest jednym z komponentów projektu distribution\cite{distribution-github}, zestawu narzędzi do pakowania, przesyłania, przechowywania i dostarczania zawartości artefaktów.

%\subsection{Prometheus}\label{subsec:prometheus}
%
%\todo{Prometheus}
%
%\subsection{Grafana}\label{subsec:grafana}
%
%\todo{Grafana}

\subsection{Medusa}\label{subsec:medusa}

Handel, w tym ten internetowy, jest obecny w naszym życiu od dekad.
Przez swoją bogatą historię stał się jedną z najlepiej opisanych i najbardziej dojrzałych domen.
Większość wyzwań i problemów, które mogą się pojawić przy implementacji sklepu internetowego zostało już świetnie udokumentowanych, chociażby w książkach z archetypami (pierwowzorami projektowymi).
W obliczu tego, obecnie niewiele sklepów internetowych jest tworzone od zera, bez podparcia gotowymi rozwiązaniami.

Część praktyczna niniejszej pracy korzysta z rozbudowanego ekosystemu Medusa\cite{medusajs-home}.
Wykorzystanie tego typu rozwiązania znacząco przyśpieszyło proces wdrożenia platformy i pozwoliło skupić się na elementach specyficznych dla tematu pracy.
Modularna architektura \emph{Software Development Kit (SDK)} projektu Medusa zawiera moduły dla każdej niezbędnej funkcji sklepu internetowego, chociażby moduł odpowiedzialny za obsługę katalogu produktów, logistykę czy płatności.

\subsection{TypeScript}\label{subsec:typescript}

\todo{TypeScript}

\subsection{Node.js}\label{subsec:nodejs}

\todo{Node.js}

\subsection{Got}\label{subsec:got}

\todo{Got}

\newpage

\section{Wyniki testów i analiza}\label{sec:testy}

\subsection{Scraper}

\subsection{Rate Limiting}

\subsection{Web Application Firewall}

\subsection{Browser Fingerprinting}

\newpage


\section{Podsumowanie}\label{sec:podsumowanie}

Praca ta rzuca światło na kwestię web scrapingu pozostawiając kilka istotnych wniosków w kontekście cyberbezpieczeństwa.
Web scraping jawi się jako atrakcyjne i efektywne narzędzie do zbierania danych.

Rekonesans okrył fakt dynamicznego renderowania treści z danymi pochodzącymi z publicznie dostępnego API\@, co pozwoliło na API web scraping.
Scraper okazał się relatywnie łatwy i szybki w implementacji.
Wywnioskowano, że takie witryny są szczególnie narażone na scraping.

Opisane i wdrożone łącznie metody detekcji web scrapingu okazały się skutecznie zablokować stworzony scraper.
Niemniej jednak, tylko dwa spośród trzech zabezpieczeń są w stanie to zrobić w pojedynkę.
Pewnym rozczarowaniem okazało się blokowanie regułowe.
Ukazano jego słabe strony i uznano, że metoda ta służy głównie do detekcji prostych i powszechnie znanych botów.
Rate Limiting, zalecany przez OWASP, ogranicza nadużycia generowane przez zautomatyzowane narzędzia.
Najbardziej złożona i efektywna okazała się metoda wykorzystująca browser fingerprinting.

Mimo tych zabezpieczeń system teoretycznie nadal może być scrapowany.
Wymagałoby to jednak znacznie większego wysiłku i determinacji od drugiej strony.
W rezultacie, proporcja kosztu do zysku może skutecznie do tego zniechęcać.

Projekt posiada potencjał do dalszego rozwoju w zakresie detekcji web scrapingu.
Dla blokowania regułowego kluczowe jest stałe utrzymywanie i dostosowywanie reguł do zmieniających się okoliczności.
Wykorzystany NGINX Ultimate Bot Blocker posiada skrypt, który pozwala na jego automatyczne aktualizacje.
Taki skrypt mógłby być uruchamiany w ustalonym interwale np.~w procesie \texttt{cron}.
W aspekcie metody ograniczającej tempo żądań, NGNIX Ingress Controller daje możliwość granularnego dostosowania i zróżnicowania reguł dla różnych końcówek API\@.
Przykładowo, końcówka z paginacją produktów mogłaby mieć inne ograniczenie niż końcówka z szczegółami produktu.
To utrudniłoby szybką enumerację po wszystkich stronach.
Ponadto, istnieje możliwość wprowadzenia kilku ograniczeń tempa żądań działających jednocześnie.
Powstały \texttt{Bot API} można wzbogacić o analizę nagłówków HTTP i SSL fingerprinting.
Ponadto, identyfikator \texttt{x-bot-id} powinien posiadać swoją ważność.

%--------------------------------------------
% Literatura
%--------------------------------------------
\cleardoublepage % Zaczynamy od nieparzystej strony
\printbibliography

%--------------------------------------------
% Spisy (opcjonalne)
%--------------------------------------------
\newpage
\pagestyle{plain}

\section*{Wykaz symboli i skrótów}
\acronym{API}{Application Programming Interface}
\acronym{CLI}{Command Line Interface}
\acronym{CPU}{Central Processing Unit}
\acronym{CSV}{Comma-Separated Values}
\acronym{DoS}{Denial Of Service}
\acronym{GPU}{Graphics Processing Unit}
\acronym{HTML}{Hypertext Markup Language}
\acronym{HTTP}{Hypertext Transfer Protocol}
\acronym{JSON}{JavaScript Object Notation}
\acronym{LB}{Load Balancer}
\acronym{NLB}{Network Load Balancer}
\acronym{OCI}{Oracle Cloud Infrastructure}
\acronym{OCPU}{The Oracle CPU}
\acronym{WWW}{World Wide Web}
\acronym{XLSX}{Microsoft Excel Open XML Spreadsheet}
\acronym{XML}{Extensible Markup Language}

\vspace{0.8cm}          % vertical space
\section*{Wykaz rysunków}
\printlist[figures]{lof}{1}{\setcounter{tocdepth}{3}}

\vspace{0.8cm}          % vertical space
\listoflistings
\captionsetup[listing]{list=no} % exclude appendices listings

\vspace{0.8cm}          % vertical space
\section*{Wykaz tabel}
\printlist[tables]{lot}{1}{\setcounter{tocdepth}{3}}

\newpage
\appendix
\appendixpage
\addappheadtotoc

\stopcontents[mainsections]
\stoplist[figures]{lof}
\stoplist[tables]{lot}

\startcontents[appendices]
\printcontents[appendices]{l}{1}{\setcounter{tocdepth}{3}}

\newpage

\section{Instrukcja stworzenia klastra Kubernetes}

Ten załącznik stanowi instrukcję stworzenia klastra Kubernetes w chmurze obliczeniowej Oracle Cloud Infrastructure (OCI).
Klaster składa się z czterech węzłów, każdy z nich posiada 1 procesor (Oracle CPU, OCPU) oraz 6 GB pamięci RAM\@.
Zaproponowana infrastruktura obejmuje cztery maszyny wirtualne (c1, c2, c3, c4), load balancer (k8s) oraz sieć wirtualną (subnet-20230313-1902).
Uproszczony i uporządkowany model infrastruktury przedstawiono na rysunku~\ref{fig:infrastructure}.
Wszystkie wymienione elementy zostały wdrożone w ramach darmowego planu \emph{Always Free}.

Wykorzystanie czterech węzłów wraz z load balancerem, oraz odpowiednia konfiguracja, pozwala na zapewnienie wysokiej dostępności (ang. \emph{high availability}).

\begin{figure}[H]
    \centering
    \includegraphics[width=\textwidth]{img/oci-infrastructure}
    \caption{Zaprojektowana infrastruktura}
    \label{fig:infrastructure}
\end{figure}

\subsection{Maszyny wirtualne}

Każdy węzeł klastra został uruchomiony na oddzielnej maszynie wirtualnej (ang. \emph{virtual machine, VM}).
Wszystkie węzły, w formie tabelki, przedstawiono na rysunku~\ref{fig:oci-compute-instances}.

\begin{figure}[H]
    \centering
    \includegraphics[width=\textwidth]{img/oci-compute-instances}
    \caption{Lista wszystkich maszyn wirtualnych}
    \label{fig:oci-compute-instances}
\end{figure}

\noindent Specyfikacja każdej z maszyn wirtualnych:
\begin{itemize}
    \item Kształt: VM.Standard.A1.Flex (Procesor Arm od Ampere)
    \item Liczba OCPU: 1
    \item Przepustowość sieci: 1 Gbps
    \item Pamięć RAM: 6 GB
    \item Obraz: Canonical Ubuntu 22.04 aarch64 2023.02.15-0
\end{itemize}

\noindent Szczegółową specyfikację jednej z maszyn wirtualnych przedstawiono na rysunku~\ref{fig:oci-instance-details}.

\begin{figure}[H]
    \centering
    \includegraphics[width=\textwidth]{img/oci-instance-details}
    \caption{Specyfikacja maszyny wirtualnej c1}
    \label{fig:oci-instance-details}
\end{figure}

\subsection{Konfiguracja podsieci}

Wszystkie zasoby znajdują się w jednej wirtualnej podsieci (ang. \emph{virtual cloud network}, VCN) subnet-20230313-1902, która stanowi bezpieczne i izolowane środowisko sieciowe.
Informacje o podsieci przedstawiono na rysunku~\ref{fig:oci-subnet}.

\begin{figure}[H]
    \centering
    \includegraphics[width=\textwidth]{img/oci-subnet}
    \caption{Podsieć subnet-20230313-1902}
    \label{fig:oci-subnet}
\end{figure}

\noindent Domyślne reguły VCN\cite{oci-security-lists} pozwalają na:
\begin{enumerate}
    \item ruch TCP na porcie usługi SSH (22) z autoryzowanych adresów IP\@,
    \item ruch ICMP typu 3 o kodzie 4 (ang. \emph{Fragmentation Needed and Don't Fragment was Set}) z dowolnego adresu IP,
    \item ruch ICMP typu 3 z wszystkich hostów znajdujących się w danej podsieci,
    \item ruch wychodzący.
\end{enumerate}

\noindent Aby wykorzystać klaster jako platformę wdrożeniową dla aplikacji internetowych, konieczne jest dodanie dwóch dodatkowych reguł.

\begin{enumerate}
    \item Reguła pozwalająca na ruch TCP z dowolnego źródła na porty 80 i 443.
    \item Reguła pozwalająca na cały ruch z adresu IP administratora - wymagana do zdalnego zarządzania klastrem przez Kubernetes API (kubectl).
\end{enumerate}

\noindent Kompletna lista wszystkich reguł sieciowych dla stworzonej podsieci subnet-20230313-1902 została zaprezentowana na rysunku~\ref{fig:oci-subnet-ingress-rules}.

\begin{figure}[H]
    \centering
    \includegraphics[width=\textwidth]{img/oci-subnet-ingress-rules}
    \caption{Reguły dla ruchu przychodzącego}
    \label{fig:oci-subnet-ingress-rules}
\end{figure}

Zmiany na poziome usług sieciowych OCI są niewystarczające, ponieważ ruch zostanie zablokowany przez domyślne reguły firewalla systemowego maszyn wirtualnych.
Aby osiągnąć oczekiwany rezultat konieczna jest modyfikacja pliku \texttt{/etc/iptables/rules.v4} na każdej z maszyn wirtualnych.

\noindent Należy dodać dwie reguły:

\begin{enumerate}
    \item Reguła zezwalająca na ruch wejściowy pochodzący z publicznego adresu IP administratora.\\
    \mintinline{text}{-I INPUT -s ADMIN_PUBLIC_IP/32 -j ACCEPT}
    \item Reguła pozwalająca na ruch przychodzący z dowolnego adresu w podsieci.\\
    \mintinline{text}{-I INPUT -s 10.0.0.0/24 -j ACCEPT}
\end{enumerate}

\noindent Dodatkowo, konieczne jest usunięcie wpisu, który blokuje wiadomości ICMP (ping):\\
\mintinline{text}{-A FORWARD -j REJECT --reject-with icmp-host-prohibited}

\subsection{Network Load Balancer}

Load Balancer (LB) to technologia, która, podobnie do reverse proxy, ma na celu kierowanie żądań użytkowników do odpowiednich serwerów.
Każdy LB implementuje pewien zestaw reguł i algorytmów, który równomiernie rozkładają ruch na wszystkie serwery w grupie mogące go obsłużyć, tak aby sprostować aktualnemu obciążeniowi.
Dzięki temu zapewnia pryncypium wysokiej dostępności (high availability), niezawodności i skalowalność systemu.
W związku z tym, że Load Balancer ciągle monitoruje aktywność węzłów, jest w stanie szybko wykryć ewentualną awarię i automatycznie przekierować ruch do pozostałych, sprawnych węzłów.
Prostą analogią dla Load Balancera jest policjant stojący na skrzyżowaniu.

Dodatkową zaletą LB jest uproszczenie konfiguracji DNS - wystarczy dodanie jednego rekord typu A z publicznym adresem IP wskazującym na LB\@.
Eliminuje to potrzebę tworzenia osobnych wpisów dla każdego węzła, co komplikowałoby konfigurację i utrudniałoby jej utrzymanie.

Z powodu zapewnienia wysokiej dostępności, równomiernego rozłożenia ruchu, skalowalności i uproszczeniu konfiguracji DNS, jak i wielu innych niewspomnianych korzyści, LB jest kluczowym elementem wielu rozbudowanych systemów.

Jednym z rodzajów Load Balancera jest Network Load Balancer (NLB), który działa na warstwie 3 i 4 modelu OSI wykorzystując protokoły TCP, UDP i ICMP\@.
NLB przekazuje pakiety do i z serwera nadrzędnego na podstawie informacji na poziomie IP, portu i protokołu bez sprawdzania pakietów.

Utworzony Network Load Balancer k8s znajduje się w wcześniej stworzonej podsieci subnet-20230313-1902.

\autoref{fig:oci-network-load-balancer-k8s} przedstawia informacje o wdrożonym NLB k8s.

\begin{figure}[H]
    \centering
    \includegraphics[width=\textwidth]{img/oci-network-load-balancer-k8s}
    \caption{Informacje o NLB k8s}
    \label{fig:oci-network-load-balancer-k8s}
\end{figure}

NLB k8s został skonfigurowany tak aby działał dla ruchu HTTP (port 80), HTTPS (port 443) oraz Kubernetes API (port 16443).
Dla każdego z rodzajów ruchu został utworzony Backend Set (zob. \autoref{fig:oci-network-load-balancer-k8s-backend-sets}) oraz Listener (zob. \autoref{fig:oci-network-load-balancer-k8s-listeners}).

\begin{figure}[H]
    \centering
    \includegraphics[width=\textwidth]{img/oci-network-load-balancer-k8s-backend-sets}
    \caption{Backend Sets stworzonego Network Load Balancera k8s}
    \label{fig:oci-network-load-balancer-k8s-backend-sets}
\end{figure}

\begin{figure}[H]
    \centering
    \includegraphics[width=\textwidth]{img/oci-network-load-balancer-k8s-listeners}
    \caption{Listeners stworzonego Network Load Balancera k8s}
    \label{fig:oci-network-load-balancer-k8s-listeners}
\end{figure}

Load Balancer regularnie monitoruje aktywność każdego węzła za pomocą zdefiniowanego testu Health Check.
Dla pierwszych dwóch Backend Sets - \texttt{ingress-http} (zob. \autoref{fig:oci-network-load-balancer-ingress-http-health-check}) oraz \texttt{ingress-https} (zob. \autoref{fig:oci-network-load-balancer-ingress-https-health-check})- test polega na wysłaniu zapytania z użyciem kolejno protokołu HTTP i HTTPS na ścieżkę \url{/}.
Oczekiwaną odpowiedzią serwera jest status 404.

\begin{multicols}{2}
    \begin{figure}[H]
        \centering
        \includegraphics[width=0.5\textwidth]{img/oci-network-load-balancer-ingress-http-health-check}
        \caption{Health Check dla ingress-http}
        \label{fig:oci-network-load-balancer-ingress-http-health-check}
    \end{figure}
    \begin{figure}[H]
        \centering
        \includegraphics[width=0.5\textwidth]{img/oci-network-load-balancer-ingress-https-health-check}
        \caption{Health Check dla ingress-https}
        \label{fig:oci-network-load-balancer-ingress-https-health-check}
    \end{figure}
\end{multicols}

Ostatni Backend Set, czyli  k8s-api, polega na wysłaniu zapytania protokołem HTTPS na port 16443 do ścieżki \url{/healthz}, gdzie oczekiwaną odpowiedzią serwera jest status 200 (zob. \autoref{fig:oci-network-load-balancer-k8s-api-health-check}).
\begin{figure}[H]
    \centering
    \includegraphics[width=0.7\textwidth]{img/oci-network-load-balancer-k8s-api-health-check}
    \caption{Health Check dla k8s-api}
    \label{fig:oci-network-load-balancer-k8s-api-health-check}
\end{figure}

\subsection{Instalacja Kubernetes}

Klaster wykorzystuje dystrybucję MicroK8s (zob. \autoref{subsec:microk8s}) w wersji 1.23.
Instalację MicroK8s przeprowadzono na każdej z maszyn wirtualnych poleceniem~\autoref{lst:microk8s-install}.

\begin{listing}[H]
    \begin{minted}{bash}
    apt update && \
        apt install docker.io -y && \
        snap install microk8s --classic --channel=1.23/stable
    \end{minted}
    \caption{Polecenie instalacyjne MicroK8s}
    \label{lst:microk8s-install}
\end{listing}

\subsection{Konfiguracja certyfikatów Kubernetes API}

Kubernetes API stanowi kluczowy element systemu z perspektywy cyberbezpieczeństwa.
Jednym z sposobów jego zabezpieczenia jest wymóg szyfrowanej komunikacji HTTPS, opartej na zweryfikowanym certyfikacie.
Domyślne generowane przez MicroK8s certyfikaty pozwalają na połączenia z lokalnych adresów IP (zwykle adresów LAN) oraz komunikację z adresami mDNS (komunikacja w obrębie klastra, z adresów takich jak kubernetes.default lub kubernetes.default.svc.cluster.local).
Aby umożliwić zdalne połączenie z Kubernetes API z internetu, niezbędna jest modyfikacja sekcji alt\_name pliku szablonu Certificate Signing Request (CSR) znajdującego się pod ścieżką \url{ /var/snap/microk8s/current/certs/csr.conf.template}.
Modyfikacja musi zostać wykonana na każdej z maszyn wirtualnych.

\noindent Modyfikacja polega na dodaniu do sekcji alt\_names:
\begin{enumerate}
    \item publicznego adresu IP maszyny wirtualnej,
    \item prywatnego adresu IP maszyny wirtualnej,
    \item publicznego adresu IP load balancera.
\end{enumerate}

\noindent\autoref{lst:domyslna-konfiguracja-alt-names} przedstawia domyślną, niezmienioną sekcję alt\_names.

\begin{listing}[H]
    \begin{minted}{text}
[ alt_names ]
DNS.1 = kubernetes
DNS.2 = kubernetes.default
DNS.3 = kubernetes.default.svc
DNS.4 = kubernetes.default.svc.cluster
DNS.5 = kubernetes.default.svc.cluster.local
IP.1 = 127.0.0.1
IP.2 = 10.152.183.1
    \end{minted}
    \caption{Domyślna sekcja alt\_names maszyny wirtualnej c1}
    \label{lst:domyslna-konfiguracja-alt-names}
\end{listing}

\noindent\autoref{lst:zmodyfikowana-konfiguracja-alt-names} przedstawia odpowiednio zmodyfikowaną sekcję alt\_names.

\begin{listing}[H]
    \begin{minted}{text}
[ alt_names ]
DNS.1 = kubernetes
DNS.2 = kubernetes.default
DNS.3 = kubernetes.default.svc
DNS.4 = kubernetes.default.svc.cluster
DNS.5 = kubernetes.default.svc.cluster.local
IP.1 = 127.0.0.1
IP.2 = 10.152.183.1
IP.100 = 130.61.138.232 # Publiczny adres IP maszyny wirtualnej
IP.110 = 10.0.0.85      # Prywanty adres IP maszyny wirtualnej
IP.120 = 130.61.217.255 # Publiczny adres IP load balancera
    \end{minted}
    \caption{Zmodyfikowana sekcja alt\_names na maszynie wirtualnej c1}
    \label{lst:zmodyfikowana-konfiguracja-alt-names}
\end{listing}

\noindent Po modyfikacji pliku konieczne jest uruchomienie polecenia odświeżającego certyfikaty MicroK8s (zob. \autoref{lst:polecenie-odswiezajace-certyfikaty}) i ponowne uruchomienie maszyny (reboot).

\begin{listing}[H]
    \begin{minted}{bash}
    sudo microk8s refresh-certs
    \end{minted}
    \caption{Polecenie odświeżające certyfikaty MicroK8s}
    \label{lst:polecenie-odswiezajace-certyfikaty}
\end{listing}

\subsection{Formowanie klastra Kubernetes}

\noindent Formowanie klastra w Kubernetes polega na zintegrowaniu węzłów uruchomionych na różnych maszynach wirtualnych, aby wspólnie tworzyły jednolity system.
Poniżej przedstawione są kroki niezbędne do uformowania klastra.

\begin{enumerate}
    \item Określ, która z maszyn wirtualnych będzie pełnić rolę głównego węzła (ang. \emph{master node}). To on będzie koordynować dodawanie kolejnych węzłów do klastra.
    \item Na wyznaczonym głównym węźle wykonaj polecenie \mintinline{bash}{microk8s add-node}.
          Wynikiem będzie polecenie do dołączenia do klastra mające postać:
    \begin{figure}[H]
        \begin{minted}{bash}
    microk8s join <vm_private_ip>:25000/<token>
        \end{minted}
        \label{fig:join-cluster-command}
    \end{figure}
    \item Wykonaj otrzymane w poprzednim kroku polecenie na każdej maszynie wirtualnej, która ma zostać częścią klastra, ale jeszcze w nim nie jest.
    \item Dla każdej kolejnej maszyny, która ma dołączyć do klastra, powtarzaj kroki 2 i 3.
\end{enumerate}

\end{document}
