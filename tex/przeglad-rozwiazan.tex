\newpage


\section{Przegląd istniejących rozwiązań}\label{sec:przeglad-rozwiazan}

\subsection{Rate Limiting}\label{subsec:rate-limiting}

Jak wskazuje \emph{OWASP API Security Top 10} w punkcie \emph{API4:2019 Lack of Resources \& Rate Limiting}:
\begin{displayquote}[\citetitle*{owasp-api-security-top-10}~\cite{owasp-api-security-top-10}, tłum. własne]
    Żądania API zużywają zasoby takie jak sieć, CPU, pamięć i miejsce na dysku.
    Ilość zasobów potrzebnych do zaspokojenia żądania w dużej mierze zależy od danych wejściowych użytkownika i logiki biznesowej koncówki.
    Należy również wziąć pod uwagę fakt, że żądania od wielu klientów API konkurują o te same zasoby.
    API jest podatne na problemy, jeśli brakuje przynajmniej jednego z następujących limitów lub są one ustawione nieodpowiednio (np. zbyt niskie/wysokie):

    \begin{itemize}
        \item Limit czasu wykonania
        \item Maksymalna alokowalna pamięć
        \item Liczba deskryptorów plików
        \item Liczba procesów
        \item Rozmiar ładunku żądania (np. przesyłane pliki)
        \item Liczba żądań na klienta/zasób
        \item Liczba rekordów na stronę zwracanych w pojedynczej odpowiedzi na żądanie
    \end{itemize}
\end{displayquote}

\noindent

\todo{Rate Limiting}

\subsection{Web Application Firewall}\label{subsec:waf}

\todo{WAF}

\subsection{Browser Fingerprinting}\label{subsec:browser-fingerprinting}

\todo{Browser Fingerprinting}
