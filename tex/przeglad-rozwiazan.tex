\newpage


\section{Przegląd metod detekcji}\label{sec:przeglad-rozwiazan}

\subsection{Rate Limiting}\label{subsec:rate-limiting}

Rate Limiting to technika kontrolująca tempo, w jakim klienci mogą wysyłać żądania do serwera API\@.
W tym celu zlicza się liczbę żądań w określonym oknie czasowym, a następnie ustala, czy ich częstotliwość nie przekracza maksymalnego dopuszczalnego progu~\cite{api-rate-limit-adoption}.

Rozwiązanie zazwyczaj polega na liczeniu czasu między każdym żądaniem z każdego adresu IP\@.
W przypadku, kiedy liczba żądań z danego adresu IP przekroczy ustalony limit w danym oknie czasowym, żądanie kończy się odpowiednim błędem.
Adres IP, w tym przypadku, służy jako identyfikator klienta~\cite{cloudflare-what-is-rate-limiting}.
Jednak, dopuszcza się również inne identyfikatory.

Brak rate limitingu API jest, według \emph{OWASP API Security Top 10}, uznawany za podatność.
W punkcie \emph{API4:2019 Lack of Resources \& Rate Limiting}, autorzy wskazują, że:\@
\begin{displayquote}[\citetitle*{owasp-api-security-top-10}~\cite{owasp-api-security-top-10}, tłum. własne]
    ``Żądania API zużywają zasoby takie jak sieć, CPU, pamięć i miejsce na dysku.
    Ilość zasobów potrzebnych do zaspokojenia żądania w dużej mierze zależy od danych wejściowych użytkownika i logiki biznesowej koncówki.
    Należy również wziąć pod uwagę fakt, że żądania od wielu klientów API konkurują o te same zasoby.
    API jest podatne na problemy, jeśli brakuje przynajmniej jednego z następujących limitów lub są one ustawione nieodpowiednio (np. zbyt niskie/wysokie):

    \begin{itemize}
        \item Limit czasu wykonania
        \item Maksymalna alokowalna pamięć
        \item Liczba deskryptorów plików
        \item Liczba procesów
        \item Rozmiar ładunku żądania (np. przesyłane pliki)
        \item Liczba żądań na klienta/zasób
        \item Liczba rekordów na stronę zwracanych w pojedynczej odpowiedzi na żądanie''
    \end{itemize}
\end{displayquote}

Rate Limiting stosuje się do ochrony ograniczonych zasobów, obrony przed atakami typu odmowa dostępu (ang. \emph{DoS, Denial of Service})
oraz blokowania aktywności botów generujących duże nadużycia API\@.

\newpage
\subsection{Systemy blokowania botów}\label{subsec:waf}

\todo{Systemy blokowania botów}

\subsection{Browser Fingerprinting}\label{subsec:browser-fingerprinting}

\todo{Browser Fingerprinting}
