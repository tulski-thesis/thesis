\newpage


\section{Podsumowanie}\label{sec:podsumowanie}

Praca ta rzuca światło na kwestię web scrapingu pozostawiając kilka istotnych wniosków w kontekście cyberbezpieczeństwa.
Web scraping jawi się jako atrakcyjne i efektywne narzędzie do zbierania danych.

Rekonesans ujawnił fakt dynamicznego renderowania treści z danymi pochodzącymi z publicznie dostępnego API\@, co pozwoliło na API web scraping.
Scraper okazał się relatywnie łatwy i szybki w implementacji.
Wywnioskowano, że takie witryny są szczególnie narażone na scraping.

Opisane i wdrożone symultanicznie metody detekcji web scrapingu skutecznie zablokowały stworzony scraper.
Niemniej jednak, tylko dwa spośród trzech zabezpieczeń są w stanie to zrobić w pojedynkę.
Pewnym rozczarowaniem okazało się blokowanie regułowe.
Ukazano jego słabe strony i uznano, że metoda ta służy głównie do detekcji prostych i powszechnie znanych botów.
Rate Limiting, zalecany przez OWASP, ogranicza nadużycia generowane przez zautomatyzowane narzędzia.
Najbardziej złożona i efektywna okazała się metoda wykorzystująca browser fingerprinting.

Mimo tych zabezpieczeń system teoretycznie nadal może być scrapowany.
Wymagałoby to jednak znacznie większego wysiłku i determinacji od drugiej strony.
W rezultacie, proporcja kosztu do zysku może skutecznie do tego zniechęcać.

Projekt posiada potencjał do dalszego rozwoju w zakresie detekcji web scrapingu.
Dla blokowania regułowego kluczowe jest stałe utrzymywanie i dostosowywanie reguł do zmieniających się okoliczności.
Wykorzystany NGINX Ultimate Bot Blocker posiada skrypt, który pozwala na jego automatyczne aktualizacje.
Skrypt ten mógłby być uruchamiany w ustalonym interwale np.~w procesie \texttt{cron}.
W aspekcie metody ograniczającej tempo żądań, NGNIX Ingress Controller daje możliwość granularnego dostosowania i zróżnicowania reguł dla różnych końcówek API\@.
Przykładowo, końcówka z paginacją produktów mogłaby mieć inne ograniczenie niż końcówka ze szczegółami produktu.
To utrudniłoby szybką enumerację wszystkich stron.
Ponadto, istnieje możliwość wprowadzenia kilku ograniczeń tempa żądań działających jednocześnie.
Powstały \texttt{Bot API} można wzbogacić o analizę nagłówków HTTP i SSL fingerprinting.
Ponadto, identyfikator \texttt{x-bot-id} powinien posiadać swoją ważność.