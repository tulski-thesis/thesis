\newpage

\section{Wykorzystane narzędzia}\label{sec:wykorzystane-narzedzia}

W tym rozdziale zaprezentowane zostaną technologie i oprogramowanie, które stanowiły podstawę do stworzenia i zarządzaniem platformy badawczej.

\subsection{Docker}\label{subsec:docker}

Docker to otwarte oprogramowanie do tworzenia, dostarczania i uruchamiania aplikacji\cite{docker-overview}.
Docker umożliwia uruchamianie aplikacji w izolowanym środowisku, które nazywane jest kontenerem.
Kontenery zapewniają izolację, bezpieczeństwo, spójność działania w różnych środowiskach oraz ułatwiają rozwój, testowanie i wdrażanie aplikacji.
Część praktyczna pracy wykorzystuje technologię konteneryzacji oferowaną przez platformę Docker.

\subsection{Kubernetes}\label{subsec:kubernetes}

Kubernetes to przenośna, rozszerzalna, otwarto-źródłowa platforma do zarządzania skonteneryzowanymi jednostkami obciążeniowymi (ang. \emph{workloads}) i usługami, która ułatwia zarówno deklaratywne konfigurowanie, jak i automatyzację\cite{kubernetes-overview}.
Platforma Kubernetes oferuje wiele funkcji, w tym deklaratywne zarządzanie elementami infrastruktury, usługi sieciowe, skalowanie oraz automatyzację wdrażania aplikacji, zapewniając tym samym niezbędne narzędzia potrzebne do budowy infrastruktury systemów informatycznych.

\subsection{MicroK8s}\label{subsec:microk8s}

Kubernetes stał się niejako standardem w branży IT\@.
Wraz z wzrostem jego popularności, kolejne osoby oraz firmy zaczęły dostosowywać oryginalny projekt do swoich potrzeb, tym samym tworząc jego własne dystrybucje.
Chociaż wszystkie te dystrybucje mają wspólny fundament, jakim jest oryginalny projekt Kubernetes, każda z nich wnosi unikalne funkcje, narzędzia i optymalizacje.
Przykładowo, dystrybucje takie jak Amazon Elastic Kubernetes Service (Amazon EKS), Google Kubernetes Engine (GKE) czy Azure Kubernetes Service (AKS) zostały specjalnie dostosowane pod specyficzne środowisko chmury poszczególnych dostawcy.

Niniejsza praca wykorzystuje dystrybucję MicroK8s\cite{microk8s-docs-home} ze względu na:
\begin{enumerate}
    \item łatwość instalacji i uruchomienia,
    \item fakt, że jest to lekka dystrybucja K8s, co przekłada się na mniejsze wymagania sprzętowe oraz mniejsze zużycie zasobów,
    \item stabilność - dystrybucja jest przygotowania do produkcyjnego uruchomienia,
    \item prostotę zarządzania - MicroK8s udostępnia rozbudowany interfejs wiersza poleceń (ang. \emph{command line interface, CLI}) ułatwiający konfigurację i utrzymanie środowiska.
\end{enumerate}

\subsection{Helm}\label{subsec:helm}

Helm to menadżer pakietów w środowisku Kubernetes, który pozwala definiować, instalować i aktualizować najbardziej skomplikowane aplikacje Kubernetes~\cite{helm-home}.
Narzędzie to jest szczególnie przydatne w przypadku aplikacji, które wymagają skomplikowanej konfiguracji i/lub składają się z wielu elementów.

\subsection{PostgreSQL}\label{subsec:postgresql}

PostgreSQL, nazywany także Postgres, to, według twórców, najbardziej zaawansowany otwarto-źródłowy system relacyjnych baz danych\cite{postgresql-home}.

\subsection{NGINX}\label{subsec:nginx}

\todo{NGINX}

\subsection{Cert-Manager}\label{subsec:cert-manager}

Cert-Manager to otwarto-źródłowy orkiestrator certyfikatów X.509 w środowiskach Kubernetes i OpenShift\cite{cert-manager-home}.
Manualny proces tworzenia certyfikatów w środowisku Kubernetes, zaprezentowany w dokumentacji\cite{kubernetes-generate-certificates-manually}, jest skomplikowany i czasochłonny.
Rozwiązaniem tego problemu jest narzędzie Cert-Manager, które go upraszcza, automatyzując proces tworzenia, zarządzania i odnawiania certyfikatów.

\subsection{Registry}\label{subsec:distribution}

Registry to narzędzie służące do przechowywania i dystrybucji obrazów kontenerów i innych artefaktów, oparte na specyfikacji OCI Distribution\cite{oci-distribution-specification-github}.
Registry jest jednym z komponentów projektu distribution\cite{distribution-github}, zestawu narzędzi do pakowania, przesyłania, przechowywania i dostarczania zawartości artefaktów.


\subsection{Prometheus}\label{subsec:prometheus}

\todo{Prometheus}

\subsection{Grafana}\label{subsec:grafana}

\todo{Grafana}

\subsection{Medusa}\label{subsec:medusa}

Handel, w tym ten internetowy, jest obecny w naszym życiu od dekad.
Przez swoją bogatą historię stał się jedną z najlepiej opisanych i najbardziej dojrzałych domen.
Większość wyzwań i problemów, które mogą się pojawić przy implementacji sklepu internetowego zostało już świetnie udokumentowanych, chociażby w książkach z archetypami (pierwowzorami projektowymi).
W obliczu tego, obecnie niewiele sklepów internetowych jest tworzone od zera, bez podparcia gotowymi rozwiązaniami.

Część praktyczna niniejszej pracy korzysta z rozbudowanego ekosystemu Medusa\cite{medusajs-home}.
Wykorzystanie tego typu rozwiązania znacząco przyśpieszyło proces wdrożenia platformy i pozwoliło skupić się na elementach specyficznych dla tematu pracy.
Modularna architektura \emph{Software Development Kit (SDK)} projektu Medusa zawiera moduły dla każdej niezbędnej funkcji sklepu internetowego, chociażby moduł odpowiedzialny za obsługę katalogu produktów, logistykę czy płatności.

\subsection{TypeScript}\label{subsec:typescript}

\subsection{Node.js}\label{subsec:nodejs}

\subsection{Got}\label{subsec:got}
