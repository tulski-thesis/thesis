\newpage

\section{Wykorzystane narzędzia}\label{sec:wykorzystane-narzedzia}

\todo{Ten rodział jest o narzędziach - o czym innym miałby być?}

\subsection{Docker}\label{subsec:docker}

Docker \todo{to narzędzie do... konteneryzacji?}
Opisywana w pracy platforma wykorzystuje technologię konteneryzacji oferowaną przez Docker.

\subsection{Kubernetes}\label{subsec:kubernetes}

\todo{Kubernetes to narzędzie do...}

\subsection{Helm}\label{subsec:helm}

Helm to narzędzie będące menadżerem pakietów w środowisku Kubernetes.
Narzędzie znacznie ułatwia proces wdrażania i utrzymywania aplikacji, szczególnie jeśli są one skomplikowane i złożone z kilku elementów.

\subsection{MicroK8s}\label{subsec:microk8s}

Kubernetes stał się niejako standardem w branży IT\@.
Wraz z wzrostem jego popularności, kolejne osoby oraz firmy zaczęły dostosowywać oryginalny projekt do swoich potrzeb, tym samym tworząc jego własne dystrybucje.
Chociaż wszystkie te dystrybucje mają wspólny fundament, jakim jest oryginalny projekt Kubernetes, każda z nich wnosi unikalne funkcje, narzędzia i optymalizacje.
Przykładowo, dystrybucje takie jak Amazon Elastic Kubernetes Service (Amazon EKS), Google Kubernetes Engine (GKE) czy Azure Kubernetes Service (AKS) zostały specjalnie dostosowane pod specyficzne środowisko chmury poszczególnych dostawcy.

Niniejsza praca wykorzystuje dystrybucję MicroK8s\cite{microk8s-docs-home} ze względu na:
\begin{enumerate}
    \item łatwość instalacji i uruchomienia,
    \item fakt, że jest to lekka dystrybucja K8s, co przekłada się na mniejsze wymagania sprzętowe oraz mniejsze zużycie zasobów,
    \item stabilność - dystrybucja jest przygotowania do produkcyjnego uruchomienia,
    \item prostotę zarządzania - MicroK8s udostępnia rozbudowany interfejs wiersza poleceń (ang. \emph{command line interface, CLI}) ułatwiający konfigurację i utrzymanie środowiska.
\end{enumerate}

\subsection{Medusa}\label{subsec:medusa}

Handel, w tym ten internetowy, jest obecny w naszym życiu od dekad.
Przez swoją bogatą historię stał się jedną z najlepiej opisanych i najbardziej dojrzałych domen.
Większość wyzwań i problemów, które mogą się pojawić przy implementacji sklepu internetowego zostało już świetnie udokumentowanych, chociażby w książkach z archetypami (pierwowzorami projektowymi).
W obliczu tego, obecnie niewiele sklepów internetowych jest tworzone od zera, bez podparcia gotowymi rozwiązaniami.

Część praktyczna niniejszej pracy korzysta z rozbudowanego ekosystemu Medusa\cite{medusajs-homepage}.
Wykorzystanie tego typu rozwiązania znacząco przyśpieszyło proces wdrożenia platformy i pozwoliło skupić się na elementach specyficznych dla tematu pracy.
Modularna architektura \emph{Software Development Kit (SDK)} projektu Medusa zawiera moduły dla każdej niezbędnej funkcji sklepu internetowego, chociażby moduł odpowiedzialny za obsługę katalogu produktów, logistykę czy płatności.
