\streszczenie
Praca koncentruje się na web scrapingu --- technice pozyskiwania informacji z internetu.
Traktuje ją z dwóch perspektyw tj.~jego przeprowadzania oraz zabezpieczania się przed nim.
Przedstawiono teoretyczne podstawy web scrapingu, jego praktyczne zastosowania oraz różne techniki przeprowadzania.
Szczegółowo opisano trzyetapowy proces: pobieranie danych, ich przetwarzanie oraz zapis i prezentację.
Szczególny nacisk położono na metody detekcji web scrapingu, w tym ograniczenie tempa żądań (rate limiting),
wykorzystanie reverse proxy z regułami blokującymi boty oraz identyfikację przy pomocy browser fingerprintingu.
Scraper oraz metody detekcji przedstawiono w postaci studium przypadku, wykorzystując do tego platformę badawczą (sklep internetowy).
Opisane i wdrożone zabezpieczenia przetestowano w kilku scenariuszach: każdą niezależnie oraz ich fuzję.
Zwrócono uwagę na znaczenie web scrapingu jako narzędzia do efektywnego pozyskiwania danych, jak również na istotę skutecznych metod jego detekcji.

\slowakluczowe web scraping, detekcja botów, rate limiting, reverse proxy, browser fingerprinting